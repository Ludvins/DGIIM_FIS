% Created 2018-04-06 vie 19:40
% Intended LaTeX compiler: pdflatex
\documentclass[11pt]{article}
\usepackage[utf8]{inputenc}
\usepackage[T1]{fontenc}
\usepackage{graphicx}
\usepackage{grffile}
\usepackage{longtable}
\usepackage{wrapfig}
\usepackage{rotating}
\usepackage[normalem]{ulem}
\usepackage{amsmath}
\usepackage{textcomp}
\usepackage{amssymb}
\usepackage{capt-of}
\usepackage{hyperref}
\author{Laura Gómez Garrido, Daniel Pozo Escalona\\
Luis Ortega Andrés, Javier Sáez Maldonado}
\date{\today}
\title{Descripción de Autores y Casos de Uso}
\hypersetup{
 pdfauthor={Laura Gómez Garrido, Daniel Pozo Escalona\\
Luis Ortega Andrés, Javier Sáez Maldonado},
 pdftitle={Descripción de Autores y Casos de Uso},
 pdfkeywords={},
 pdfsubject={},
 pdfcreator={Emacs 25.3.1 (Org mode 9.1.2)}, 
 pdflang={English}}
\begin{document}

\maketitle
\tableofcontents




\section{Casos de Uso}
\label{sec:org46092c4}

\begin{center}
\begin{tabular}{ll}
\hline
Caso de Uso & Caso1                  \vline ID\\
\hline
Actores & < Participantes>(iniciador)\\
\hline
Tipo & (primario,secundario,opcional)(esencial,real)\\
\hline
Referencias & (Requisitos)\\
\hline
Precondicion & Condiciones para que se pueda realizar el CU\\
\hline
Postcondicion & Efectos que tiene la realizacion del CU\\
\hline
Autor & <esta linea se podria repetir para mantener una historia de cambios del CU \vline Fecha \vline version \vline\\
\hline
\end{tabular}
\end{center}


\begin{center}
\begin{tabular}{l}
\hline
Proposito\\
\hline
Descripcion general del CU en 1 linea\\
\hline
\end{tabular}
\end{center}


\begin{center}
\begin{tabular}{l}
\hline
Resumen\\
\hline
<Descripcion de alto nivel del flujo normal del caso de uso\\
\hline
\end{tabular}
\end{center}

\begin{center}
\begin{tabular}{rlrl}
\hline
Curso normal &  &  & \\
\hline
1 & Actor 1: accion que hace &  & \\
\hline
2 & Actor 2: accion que hace & 3 & accion que hace el sistema\\
\hline
 &  &  & \\
\hline
\end{tabular}
\end{center}


\begin{center}
\begin{tabular}{ll}
\hline
Cursos Alternos & \\
\hline
1a & \\
\hline
1b & \\
\hline
\end{tabular}
\end{center}


\begin{center}
\begin{tabular}{llll}
\hline
Otros Datos &  & Rendimiento & \\
\hline
Frecuencia Esperada &  & Urgencia & \\
\hline
Importancia &  & Estabilidad & \\
\hline
\end{tabular}
\end{center}


\begin{itemize}
\item Comentarios:
\end{itemize}

\noindent\rule{\textwidth}{0.5pt}








\begin{center}
\begin{tabular}{ll}
\hline
Caso de Uso & Ingreso de paciente en cama/habitación      \vline CU1\\
\hline
Actores & Cliente, Personal administrativo\\
\hline
Tipo & Primario\\
\hline
Referencias & Habitaciones libres\\
\hline
Precondicion & Se necesita que el paciente tenga una enfermedad y haya una habitación libre\\
\hline
Postcondicion & El paciente tendrá una habitación asignada\\
\hline
Autor & Sistema\\
\hline
\end{tabular}
\end{center}


\begin{center}
\begin{tabular}{l}
\hline
Proposito\\
\hline
Que el paciente tenga una sala en la que ser atendido\\
\hline
\end{tabular}
\end{center}


\begin{center}
\begin{tabular}{l}
\hline
Resumen\\
\hline
El sistema buscará una sala libre para el paciente y asignará a la sala el nombre de este paciente\\
\hline
\end{tabular}
\end{center}

\begin{center}
\begin{tabular}{rlll}
\hline
Curso normal &  &  & \\
\hline
1 & Cliente: Solicita una sala &  & \\
\hline
2 & Personal administrativo:Comprueba que necesita una sala &  & \\
\hline
3 & Personal administrativo: Indica al sistema que asigne una sala & 4 & Sistema: Busca una sala\\
\hline
 &  & 5 & Sistema: Asigna una sala\\
\hline
\end{tabular}
\end{center}


\begin{center}
\begin{tabular}{ll}
\hline
Cursos Alternos & \\
\hline
2a & Personal administrativo: El cliente no necesita una sala\\
\hline
2a.2 & Personal administrativo: Indica al cliente que no está enfermo\\
\hline
\end{tabular}
\end{center}

\begin{center}
\begin{tabular}{llll}
\hline
Otros Datos &  & Rendimiento & Alto\\
\hline
Frecuencia Esperada & Alta & Urgencia & Media\\
\hline
Importancia & Alta & Estabilidad & Alta\\
\hline
\end{tabular}
\end{center}





\begin{center}
\begin{tabular}{ll}
\hline
Caso de Uso & Modificar la sala en la que está un paciente                 \vline CU2\\
\hline
Actores & Personal administrativo\\
\hline
Tipo & Secundario\\
\hline
Referencias & \\
\hline
Precondicion & Se necesita que el paciente esté en una sala\\
\hline
Postcondicion & El paciente estará en una sala diferente a la anterior\\
\hline
Autor & Personal administrativo\\
\hline
\end{tabular}
\end{center}


\begin{center}
\begin{tabular}{l}
\hline
Proposito\\
\hline
El propósito es dejar una sala libre y ocupar otra\\
\hline
\end{tabular}
\end{center}


\begin{center}
\begin{tabular}{l}
\hline
Resumen\\
\hline
Se solicitará por parte del paciente o del médico que se cambie al paciente de sala. Entonces, el sistema buscará otra y se la asignará.\\
\hline
\end{tabular}
\end{center}

\begin{center}
\begin{tabular}{rlrl}
\hline
Curso normal &  &  & \\
\hline
1 & Cliente/Médico: Solicita nueva sala &  & \\
\hline
2 & Personal administrativo: Pide al sistema nueva sala & 3 & Sistema: Asigna nueva sala\\
\hline
 &  & 4 & Sistema: Comunica nueva sala\\
\hline
5 & Personal administrativo: Comunica al cliente/médico la nueva sala &  & \\
\hline
\end{tabular}
\end{center}



\begin{center}
\begin{tabular}{ll}
\hline
Cursos Alternos & \\
\hline
3a & Sistema:No hay salas disponibles\\
\hline
4b & Sistema: comunica que no hay salas disponibles\\
\hline
\end{tabular}
\end{center}


\begin{center}
\begin{tabular}{llll}
\hline
Otros Datos &  & Rendimiento & Alto\\
\hline
Frecuencia Esperada & Baja & Urgencia & Media\\
\hline
Importancia & Media & Estabilidad & Alta\\
\hline
\end{tabular}
\end{center}


\begin{itemize}
\item Comentarios:
\end{itemize}
Si es el paciente el que solicita cambiar de sala, el personal administrativo deberá consultarle al paciente los motivos por los que solicita el cambio de sala y comprobar si son motivos suficientes para el cambio.
\end{document}
