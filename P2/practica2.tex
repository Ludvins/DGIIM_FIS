\documentclass[10pt,a4paper,spanish]{report}

\usepackage[spanish]{babel}
\usepackage[utf8]{inputenc}
\usepackage{amsmath, amsthm}
\usepackage{amsfonts, amssymb, latexsym}
\usepackage{enumerate}
\usepackage[official]{eurosym}
\usepackage{graphicx}
\usepackage[usenames, dvipsnames]{color}
\usepackage{colortbl}
\usepackage{multirow}
\usepackage{fancyhdr}
\usepackage{fancybox}
\usepackage{pseudocode}
\usepackage[all]{xy}
\usepackage{hyperref}
\usepackage{tikz}
\usepackage{pgfplots}

\pgfplotsset{compat=1.5}

% a4large.sty -- fill an A4 (210mm x 297mm) page
% Note: 1 inch = 25.4 mm = 72.27 pt
%       1 pt = 3.5 mm (approx)

% vertical page layout -- one inch margin top and bottom
\topmargin      0 mm    % top margin less 1 inch
\headheight     0 mm    % height of box containing the head
\headsep       10 mm    % space between the head and the body of the page
\textheight   250 mm
\footskip      14 mm    % distance from bottom of body to bottom of foot

% horizontal page layout -- one inch margin each side
%\oddsidemargin    0   mm    % inner margin less one inch on odd pages
%\evensidemargin   0   mm    % inner margin less one inch on even pages
%\textwidth      159.2 mm    % normal width of text on page

\usepackage[math]{iwona}
\usepackage[T1]{fontenc}
\usepackage{inconsolata}

\definecolor{webgreen}{rgb}{0, 0.5, 0} % less intense green 
\definecolor{webblue}{rgb}{0, 0, 0.5}  % less intense blue
\definecolor{webred}{rgb}{0.5, 0, 0}   % less intense red
\definecolor{dblackcolor}{rgb}{0.0,0.0,0.0}
\definecolor{dbluecolor}{rgb}{.01,.02,0.7}
\definecolor{dredcolor}{rgb}{0.8,0,0}
\definecolor{dgraycolor}{rgb}{0.30,0.3,0.30}

\newcommand{\HRule}{\rule{\linewidth}{0.5mm}}

\pagestyle{fancy}

\renewcommand{\chaptermark}[1]{%
\markboth{#1}{}}
\renewcommand{\sectionmark}[1]{%
\markright{\thesection\ #1}}
\fancyhf{}
\fancyhead[LE,RO]{\bfseries\thepage}
\fancyhead[LO]{\bfseries\leftmark}
\renewcommand{\headrulewidth}{0.5pt}
\renewcommand{\footrulewidth}{0pt}
\addtolength{\headheight}{0.5pt}
\fancypagestyle{plain}{
\fancyhead{}
\renewcommand{\headrulewidth}{0pt}
}

\usepackage{sectsty}
\chapterfont{\fontfamily{pag}\selectfont} %% for chapter if you want
\sectionfont{\fontfamily{pag}\selectfont}
\subsectionfont{\fontfamily{pag}\selectfont}
\subsubsectionfont{\fontfamily{pag}\selectfont}

\renewcommand{\labelenumi}{\arabic{enumi}. }
\renewcommand{\labelenumii}{\labelenumi\alph{enumii}) }
\renewcommand{\labelenumiii}{\labelenumii\roman{enumiii}: }
\begin{document}



\section{Casos de Uso}
\label{sec:org46092c4}

\begin{center}
\begin{tabular}{ll}
\hline
Caso de Uso & Caso1                  \vline ID\\
\hline
Actores & < Participantes>(iniciador)\\
\hline
Tipo & (primario,secundario,opcional)(esencial,real)\\
\hline
Referencias & (Requisitos)\\
\hline
Precondicion & Condiciones para que se pueda realizar el CU\\
\hline
Postcondicion & Efectos que tiene la realizacion del CU\\
\hline
Autor & <esta linea se podria repetir para mantener una historia de cambios del CU \vline Fecha \vline version \vline\\
\hline
\end{tabular}
\end{center}


\begin{center}
\begin{tabular}{l}
\hline
Proposito\\
\hline
Descripcion general del CU en 1 linea\\
\hline
\end{tabular}
\end{center}


\begin{center}
\begin{tabular}{l}
\hline
Resumen\\
\hline
<Descripcion de alto nivel del flujo normal del caso de uso\\
\hline
\end{tabular}
\end{center}

\begin{center}
\begin{tabular}{rlrl}
\hline
Curso normal &  &  & \\
\hline
1 & Actor 1: accion que hace &  & \\
\hline
2 & Actor 2: accion que hace & 3 & accion que hace el sistema\\
\hline
 &  &  & \\
\hline
\end{tabular}
\end{center}


\begin{center}
\begin{tabular}{ll}
\hline
Cursos Alternos & \\
\hline
1a & \\
\hline
1b & \\
\hline
\end{tabular}
\end{center}


\begin{center}
\begin{tabular}{llll}
\hline
Otros Datos &  & Rendimiento & \\
\hline
Frecuencia Esperada &  & Urgencia & \\
\hline
Importancia &  & Estabilidad & \\
\hline
\end{tabular}
\end{center}


\begin{itemize}
\item Comentarios:
\end{itemize}

\noindent\rule{\textwidth}{0.5pt}








\begin{center}
\begin{tabular}{ll}
\hline
Caso de Uso & Ingreso de paciente en cama/habitación      \vline CU1\\
\hline
Actores & Cliente, Personal administrativo\\
\hline
Tipo & Primario\\
\hline
Referencias & Habitaciones libres\\
\hline
Precondicion & Se necesita que el paciente tenga una enfermedad y haya una habitación libre\\
\hline
Postcondicion & El paciente tendrá una habitación asignada\\
\hline
Autor & Sistema\\
\hline
\end{tabular}
\end{center}


\begin{center}
\begin{tabular}{l}
\hline
Proposito\\
\hline
Que el paciente tenga una sala en la que ser atendido\\
\hline
\end{tabular}
\end{center}


\begin{center}
\begin{tabular}{l}
\hline
Resumen\\
\hline
El sistema buscará una sala libre para el paciente y asignará a la sala el nombre de este paciente\\
\hline
\end{tabular}
\end{center}

\begin{center}
\begin{tabular}{rlll}
\hline
Curso normal &  &  & \\
\hline
1 & Cliente: Solicita una sala &  & \\
\hline
2 & Personal administrativo:Comprueba que necesita una sala &  & \\
\hline
3 & Personal administrativo: Indica al sistema que asigne una sala & 4 & Sistema: Busca una sala\\
\hline
 &  & 5 & Sistema: Asigna una sala\\
\hline
\end{tabular}
\end{center}


\begin{center}
\begin{tabular}{ll}
\hline
Cursos Alternos & \\
\hline
2a & Personal administrativo: El cliente no necesita una sala\\
\hline
2a.2 & Personal administrativo: Indica al cliente que no está enfermo\\
\hline
\end{tabular}
\end{center}

\begin{center}
\begin{tabular}{llll}
\hline
Otros Datos &  & Rendimiento & Alto\\
\hline
Frecuencia Esperada & Alta & Urgencia & Media\\
\hline
Importancia & Alta & Estabilidad & Alta\\
\hline
\end{tabular}
\end{center}





\begin{center}
\begin{tabular}{ll}
\hline
Caso de Uso & Modificar la sala en la que está un paciente                 \vline CU2\\
\hline
Actores & Personal administrativo\\
\hline
Tipo & Secundario\\
\hline
Referencias & \\
\hline
Precondicion & Se necesita que el paciente esté en una sala\\
\hline
Postcondicion & El paciente estará en una sala diferente a la anterior\\
\hline
Autor & Personal administrativo\\
\hline
\end{tabular}
\end{center}


\begin{center}
\begin{tabular}{l}
\hline
Proposito\\
\hline
El propósito es dejar una sala libre y ocupar otra\\
\hline
\end{tabular}
\end{center}


\begin{center}
\begin{tabular}{l}
\hline
Resumen\\
\hline
Se solicitará por parte del paciente o del médico que se cambie al paciente de sala. Entonces, el sistema buscará otra y se la asignará.\\
\hline
\end{tabular}
\end{center}

\begin{center}
\begin{tabular}{rlrl}
\hline
Curso normal &  &  & \\
\hline
1 & Cliente/Médico: Solicita nueva sala &  & \\
\hline
2 & Personal administrativo: Pide al sistema nueva sala & 3 & Sistema: Asigna nueva sala\\
\hline
 &  & 4 & Sistema: Comunica nueva sala\\
\hline
5 & Personal administrativo: Comunica al cliente/médico la nueva sala &  & \\
\hline
\end{tabular}
\end{center}



\begin{center}
\begin{tabular}{ll}
\hline
Cursos Alternos & \\
\hline
3a & Sistema:No hay salas disponibles\\
\hline
4b & Sistema: comunica que no hay salas disponibles\\
\hline
\end{tabular}
\end{center}


\begin{center}
\begin{tabular}{llll}
\hline
Otros Datos &  & Rendimiento & Alto\\
\hline
Frecuencia Esperada & Baja & Urgencia & Media\\
\hline
Importancia & Media & Estabilidad & Alta\\
\hline
\end{tabular}
\end{center}


\begin{itemize}
\item Comentarios:
\end{itemize}
Si es el paciente el que solicita cambiar de sala, el personal administrativo deberá consultarle al paciente los motivos por los que solicita el cambio de sala y comprobar si son motivos suficientes para el cambio.


		\begin{tabular}{|>{\raggedright}p{58pt}|>{\raggedright}p{109pt}|>{\raggedright}p{1pt}|>{\raggedright}p{17pt}|>{\raggedright}p{28pt}|>{\raggedright}p{0pt}|>{\raggedright}p{18pt}|>{\raggedright}p{20pt}|}
	\hline
	 \textbf{Caso de Uso} &

	%Nombre del CU e identificador
	\multicolumn{5}{p{155pt}|}{Modificacion y borrado de usuarios del sistema}	& \multicolumn{2}{p{39pt}|}{\textbf{34}}\tabularnewline

	\hline

	%Listado de actores
	\textbf{Actores} & \multicolumn{7}{p{194pt}|}{}\tabularnewline
	\hline

	%Tipo de CU
	\textbf{Tipo} & \multicolumn{7}{p{194pt}|}{Tipo del caso de uso Primario, Secundario u Opcional \textbar{} Esencial o Real}\tabularnewline
	\hline

	%Requisitos del CU
	\textbf{Referencias} & \multicolumn{2}{p{110pt}|}{Creacion de nuevos usuarios en el sistema y asignacion de permisos a usuarios del sistema} & \multicolumn{5}{p{84pt}|}{}\tabularnewline
	\hline

	%Precondiciones
	\textbf{Precondición} & \multicolumn{7}{p{194pt}|}{Se debe haber realizado un cambio en la situacion de trabajo de un usuario para su posible modificación.}\tabularnewline
	\hline

	%Postcondiciones
	\textbf{Postcondición} & \multicolumn{7}{p{194pt}|}{Ante un cambio de un usuario, se deberan establecer los nuevos permisos de este.}\tabularnewline
	\hline

	%Autor y fecha
	\textbf{Autor} & Grupo 9 & \multicolumn{2}{p{30pt}|}{
	\textbf{Fecha}} & 08-04 & \multicolumn{2}{p{30pt}|}{
	\textbf{Versión}} & 1.0 \tabularnewline
	\hline
	\end{tabular}

	\vspace{0.5cm}

	%Proposito
	\begin{tabular}{|>{\raggedright}p{337pt}|}
		\hline
		\textbf{Proposito} \tabularnewline \hline
			Tener un control de los usuarios que se encuentran en el sistema.
		\tabularnewline
		\hline
	\end{tabular}

	\vspace{0.5cm}
	%Resumen
	\begin{tabular}{|>{\raggedright}p{337pt}|}
		\hline
		\textbf{Resumen}\tabularnewline
		\hline
			El sistema debe gestionar la creación o borrado de un usuario cuando sea necesario, asi como ser capaz de modificar un usuario ya existente en el sistema.
		\tabularnewline
		\hline
	\end{tabular}
	\vspace{0.5cm}
	
		\begin{tabular}{|>{\raggedright}p{58pt}|>{\raggedright}p{109pt}|>{\raggedright}p{1pt}|>{\raggedright}p{17pt}|>{\raggedright}p{28pt}|>{\raggedright}p{0pt}|>{\raggedright}p{18pt}|>{\raggedright}p{20pt}|}
	\hline
	 \textbf{Caso de Uso} &

	%Nombre del CU e identificador
	\multicolumn{5}{p{155pt}|}{Emision de facturas a los pacientes}	& \multicolumn{2}{p{39pt}|}{\textbf{35}}\tabularnewline

	\hline

	%Listado de actores
	\textbf{Actores} & \multicolumn{7}{p{194pt}|}{Pacientes}\tabularnewline
	\hline

	%Tipo de CU
	\textbf{Tipo} & \multicolumn{7}{p{194pt}|}{Primario}\tabularnewline
	\hline

	%Requisitos del CU
	\textbf{Referencias} & \multicolumn{2}{p{110pt}|}{Registro de cobros de las facturas y envio de recordatorio de pago de facturas.} & \multicolumn{5}{p{84pt}|}{Registro de cobros y pagos de las facturas.}\tabularnewline
	\hline

	%Precondiciones
	\textbf{Precondición} & \multicolumn{7}{p{194pt}|}{Se debe haber realizado el pago por parte del paciente de las facturas pertinentes.}\tabularnewline
	\hline

	%Postcondiciones
	\textbf{Postcondición} & \multicolumn{7}{p{194pt}|}{Registro del pago, no se deben enviar mas recordatorios de pago una vez realizado este.}\tabularnewline
	\hline

	%Autor y fecha
	\textbf{Autor} & Grupo 9 & \multicolumn{2}{p{30pt}|}{
	\textbf{Fecha}} & 08-04 & \multicolumn{2}{p{30pt}|}{
	\textbf{Versión}} & 1.0 \tabularnewline
	\hline
	\end{tabular}

	\vspace{0.5cm}

	%Proposito
	\begin{tabular}{|>{\raggedright}p{337pt}|}
		\hline
		\textbf{Proposito} \tabularnewline \hline
			Administrar el pago de las facturas enviadas a los pacientes.
		\tabularnewline
		\hline
	\end{tabular}

	\vspace{0.5cm}
	%Resumen
	\begin{tabular}{|>{\raggedright}p{337pt}|}
		\hline
		\textbf{Resumen}\tabularnewline
		\hline
			El sistema debe encargarse de administrar y registrar el pago de las facturas de los pacientes, además, se encargara del envio de recordatorios en caso de que el pago no se haya realizado pasada la fecha asignada.
		\tabularnewline
		\hline
	\end{tabular}
	\vspace{0.5cm}
	
	

	\begin{tabular}{|>{\raggedright}p{58pt}|>{\raggedright}p{109pt}|>{\raggedright}p{1pt}|>{\raggedright}p{17pt}|>{\raggedright}p{28pt}|>{\raggedright}p{0pt}|>{\raggedright}p{18pt}|>{\raggedright}p{20pt}|}
	\hline
	 \textbf{Caso de Uso} &

	%Nombre del CU e identificador
	\multicolumn{5}{p{155pt}|}{Pago de facturas a proveedores}	& \multicolumn{2}{p{39pt}|}{\textbf{36}}\tabularnewline

	\hline

	%Listado de actores
	\textbf{Actores} & \multicolumn{7}{p{194pt}|}{Proveedores}\tabularnewline
	\hline

	%Tipo de CU
	\textbf{Tipo} & \multicolumn{7}{p{194pt}|}{Real}\tabularnewline
	\hline

	%Requisitos del CU
	\textbf{Referencias} & \multicolumn{2}{p{110pt}|}{Registro de pago de facturas} & \multicolumn{5}{p{84pt}|}{Registor de cobro y pago de las facturas}\tabularnewline
	\hline

	%Precondiciones
	\textbf{Precondición} & \multicolumn{7}{p{194pt}|}{Se debe haber realizado un pedido a un proveedor, el pedido debe ser correcto.}\tabularnewline
	\hline

	%Postcondiciones
	\textbf{Postcondición} & \multicolumn{7}{p{194pt}|}{Se debe ajustar el capital tras realizar el pago de la factura.}\tabularnewline
	\hline

	%Autor y fecha
	\textbf{Autor} & Grupo 9 & \multicolumn{2}{p{30pt}|}{
	\textbf{Fecha}} & 08-04 & \multicolumn{2}{p{30pt}|}{
	\textbf{Versión}} & 1.0 \tabularnewline
	\hline
	\end{tabular}

	\vspace{0.5cm}

	%Proposito
	\begin{tabular}{|>{\raggedright}p{337pt}|}
		\hline
		\textbf{Proposito} \tabularnewline \hline
			Realizar el pago de los suministros y productos adquiridos de los proveedores.
		\tabularnewline
		\hline
	\end{tabular}

	\vspace{0.5cm}
	%Resumen
	\begin{tabular}{|>{\raggedright}p{337pt}|}
		\hline
		\textbf{Resumen}\tabularnewline
		\hline
			Se han de pagar las facturas pendientes con los proveedores por su servicio prestado al reabastecer los suministros del centro, además, se debe llevar un registro de estos pagos.
		\tabularnewline
		\hline
	\end{tabular}
	\vspace{0.5cm}

	
	
	
	\begin{tabular}{|>{\raggedright}p{58pt}|>{\raggedright}p{109pt}|>{\raggedright}p{1pt}|>{\raggedright}p{17pt}|>{\raggedright}p{28pt}|>{\raggedright}p{0pt}|>{\raggedright}p{18pt}|>{\raggedright}p{20pt}|}
	\hline
	 \textbf{Caso de Uso} &

	%Nombre del CU e identificador
	\multicolumn{5}{p{155pt}|}{Control de stock, inventario y consumo}	& \multicolumn{2}{p{39pt}|}{\textbf{37}}\tabularnewline

	\hline

	%Listado de actores
	\textbf{Actores} & \multicolumn{7}{p{194pt}|}{}\tabularnewline
	\hline

	%Tipo de CU
	\textbf{Tipo} & \multicolumn{7}{p{194pt}|}{Real}\tabularnewline
	\hline

	%Requisitos del CU
	\textbf{Referencias} & \multicolumn{2}{p{110pt}|}{Registro de los suministros que hay que adquirir} & \multicolumn{5}{p{84pt}|}{}\tabularnewline
	\hline

	%Precondiciones
	\textbf{Precondición} & \multicolumn{7}{p{194pt}|}{Es necesario que se haya producido el uso o consumo de un suministro.}\tabularnewline
	\hline

	%Postcondiciones
	\textbf{Postcondición} & \multicolumn{7}{p{194pt}|}{La necesidad readquirir aquellos suministros utilizados.}\tabularnewline
	\hline

	%Autor y fecha
	\textbf{Autor} & Grupo 9 & \multicolumn{2}{p{30pt}|}{
	\textbf{Fecha}} & 08-04 & \multicolumn{2}{p{30pt}|}{
	\textbf{Versión}} & 1.0 \tabularnewline
	\hline
	\end{tabular}

	\vspace{0.5cm}

	%Proposito
	\begin{tabular}{|>{\raggedright}p{337pt}|}
		\hline
		\textbf{Proposito} \tabularnewline \hline
			Tener el control total de los suministros y el inventario utilizado en el centro.
		\tabularnewline
		\hline
	\end{tabular}

	\vspace{0.5cm}
	%Resumen
	\begin{tabular}{|>{\raggedright}p{337pt}|}
		\hline
		\textbf{Resumen}\tabularnewline
		\hline
			Se debe saber en todo momento y detalle el inventario del centro. Ante el uso de un suministro o la falta de alguno, el sistema debe ser capaz de registrar la necesidad de adquirir mas.
		\tabularnewline
		\hline
	\end{tabular}
	\vspace{0.5cm}

	
	
	\begin{tabular}{|>{\raggedright}p{58pt}|>{\raggedright}p{109pt}|>{\raggedright}p{1pt}|>{\raggedright}p{17pt}|>{\raggedright}p{28pt}|>{\raggedright}p{0pt}|>{\raggedright}p{18pt}|>{\raggedright}p{20pt}|}
	\hline
	 \textbf{Caso de Uso} &

	%Nombre del CU e identificador
	\multicolumn{5}{p{155pt}|}{Asignacion de turnos y guardias}	& \multicolumn{2}{p{39pt}|}{\textbf{38}}\tabularnewline

	\hline

	%Listado de actores
	\textbf{Actores} & \multicolumn{7}{p{194pt}|}{Personal}\tabularnewline
	\hline

	%Tipo de CU
	\textbf{Tipo} & \multicolumn{7}{p{194pt}|}{Primario}\tabularnewline
	\hline

	%Requisitos del CU
	\textbf{Referencias} & \multicolumn{2}{p{110pt}|}{Asignación de turnos para el personal y asignación de guardias} & \multicolumn{5}{p{84pt}|}{}\tabularnewline
	\hline

	%Precondiciones
	\textbf{Precondición} & \multicolumn{7}{p{194pt}|}{El personal a asignar debe estar disponible en el sistema.}\tabularnewline
	\hline

	%Postcondiciones
	\textbf{Postcondición} & \multicolumn{7}{p{194pt}|}{Una vez asignado un turno o guardia, no se puede asignar otro a la misma hora.}\tabularnewline
	\hline

	%Autor y fecha
	\textbf{Autor} & Grupo 9 & \multicolumn{2}{p{30pt}|}{
	\textbf{Fecha}} & 08-04 & \multicolumn{2}{p{30pt}|}{
	\textbf{Versión}} & 1.0 \tabularnewline
	\hline
	\end{tabular}

	\vspace{0.5cm}

	%Proposito
	\begin{tabular}{|>{\raggedright}p{337pt}|}
		\hline
		\textbf{Proposito} \tabularnewline \hline
			Organizar los horarios de trabajo y de guardias de cada uno de los integrantes del personal.
		\tabularnewline
		\hline
	\end{tabular}

	\vspace{0.5cm}
	%Resumen
	\begin{tabular}{|>{\raggedright}p{337pt}|}
		\hline
		\textbf{Resumen}\tabularnewline
		\hline
			El sistema debe poder organizar los horarios necesarios, tanto regulares como de guardias, teniendo en cuenta que el msimo personal no puede tener dos turnos asignados a la misma hora.
		\tabularnewline
		\hline
	\end{tabular}
	\vspace{0.5cm}

	
	\begin{tabular}{|>{\raggedright}p{58pt}|>{\raggedright}p{109pt}|>{\raggedright}p{1pt}|>{\raggedright}p{17pt}|>{\raggedright}p{28pt}|>{\raggedright}p{0pt}|>{\raggedright}p{18pt}|>{\raggedright}p{20pt}|}
	\hline
	 \textbf{Caso de Uso} &

	%Nombre del CU e identificador
	\multicolumn{5}{p{155pt}|}{Control de las vacaciones del personal}	& \multicolumn{2}{p{39pt}|}{\textbf{39}}\tabularnewline

	\hline

	%Listado de actores
	\textbf{Actores} & \multicolumn{7}{p{194pt}|}{Personal}\tabularnewline
	\hline

	%Tipo de CU
	\textbf{Tipo} & \multicolumn{7}{p{194pt}|}{Esencial}\tabularnewline
	\hline

	%Requisitos del CU
	\textbf{Referencias} & \multicolumn{2}{p{110pt}|}{} & \multicolumn{5}{p{84pt}|}{Asignacion de guardias y de turnos}\tabularnewline
	\hline

	%Precondiciones
	\textbf{Precondición} & \multicolumn{7}{p{194pt}|}{Haber trabajado las horas estipuladas necesarias para disponer de dias libres.}\tabularnewline
	\hline

	%Postcondiciones
	\textbf{Postcondición} & \multicolumn{7}{p{194pt}|}{No se deben poder asignar turnos o guardias a aquel personal que se encuentre en periodo vacacional.}\tabularnewline
	\hline

	%Autor y fecha
	\textbf{Autor} & Grupo 9 & \multicolumn{2}{p{30pt}|}{
	\textbf{Fecha}} & 08-04 & \multicolumn{2}{p{30pt}|}{
	\textbf{Versión}} & 1.0 \tabularnewline
	\hline
	\end{tabular}

	\vspace{0.5cm}

	%Proposito
	\begin{tabular}{|>{\raggedright}p{337pt}|}
		\hline
		\textbf{Proposito} \tabularnewline \hline
			Proporcionar al personal de un periodo de descanso de sus funciones.
		\tabularnewline
		\hline
	\end{tabular}

	\vspace{0.5cm}
	%Resumen
	\begin{tabular}{|>{\raggedright}p{337pt}|}
		\hline
		\textbf{Resumen}\tabularnewline
		\hline
			Facilitar a todo integrante del personal un periodo vacacional de descanso.
		\tabularnewline
		\hline
	\end{tabular}
	\vspace{0.5cm}

	
	
	\begin{tabular}{|>{\raggedright}p{58pt}|>{\raggedright}p{109pt}|>{\raggedright}p{1pt}|>{\raggedright}p{17pt}|>{\raggedright}p{28pt}|>{\raggedright}p{0pt}|>{\raggedright}p{18pt}|>{\raggedright}p{20pt}|}
	\hline
	 \textbf{Caso de Uso} &

	%Nombre del CU e identificador
	\multicolumn{5}{p{155pt}|}{Pago al personal}	& \multicolumn{2}{p{39pt}|}{\textbf{40}}\tabularnewline

	\hline

	%Listado de actores
	\textbf{Actores} & \multicolumn{7}{p{194pt}|}{Personal a ser pagado}\tabularnewline
	\hline

	%Tipo de CU
	\textbf{Tipo} & \multicolumn{7}{p{194pt}|}{Tipo del caso de uso Primario, Secundario u Opcional \textbar{} Esencial o Real}\tabularnewline
	\hline

	%Requisitos del CU
	\textbf{Referencias} & \multicolumn{2}{p{110pt}|}{Registro de pago al personal} & \multicolumn{5}{p{84pt}|}{}\tabularnewline
	\hline

	%Precondiciones
	\textbf{Precondición} & \multicolumn{7}{p{194pt}|}{El personal correspondiente ha realizado correctamente el trabajo, y no se ha producido ningún incidente .}\tabularnewline
	\hline

	%Postcondiciones
	\textbf{Postcondición} & \multicolumn{7}{p{194pt}|}{Todo el personal ha recibido el salario correspondiente.}\tabularnewline
	\hline

	%Autor y fecha
	\textbf{Autor} &  Grupo 9 & \multicolumn{2}{p{30pt}|}{
	\textbf{Fecha}} & 08-04 & \multicolumn{2}{p{30pt}|}{
	\textbf{Versión}} & 1.0 \tabularnewline
	\hline
	\end{tabular}

	\vspace{0.5cm}

	%Proposito
	\begin{tabular}{|>{\raggedright}p{337pt}|}
		\hline
		\textbf{Proposito} \tabularnewline \hline
	Realizar el pago correspondiente al indicado en su contrato a cada integrante del personal.
		\tabularnewline
		\hline
	\end{tabular}

	\vspace{0.5cm}
	%Resumen
	\begin{tabular}{|>{\raggedright}p{337pt}|}
		\hline
		\textbf{Resumen}\tabularnewline
		\hline
		El sistema debe ser capaz de realizar el pago a cada uno de los integrantes del personal, teniendo en cuenta los diferentes salarios de cada uno.
		Además debe dejar registro del pago realizado.
		\tabularnewline
		\hline
	\end{tabular}
	\vspace{0.5cm}
	


\end{document}
