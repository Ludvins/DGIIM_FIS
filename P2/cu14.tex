\begin{tabular}{|>{\raggedright}p{58pt}|>{\raggedright}p{109pt}|>{\raggedright}p{1pt}|>{\raggedright}p{17pt}|>{\raggedright}p{28pt}|>{\raggedright}p{0pt}|>{\raggedright}p{18pt}|>{\raggedright}p{20pt}|}
	\hline
	 \textbf{Caso de Uso} &

	%Nombre del CU e identificador
	\multicolumn{5}{p{155pt}|}{Inserción de datos sobre una prueba}	& \multicolumn{2}{p{39pt}|}{\textbf{14}}\tabularnewline

	\hline

	%Listado de actores
	\textbf{Actores} & \multicolumn{7}{p{194pt}|}{Personal médico}\tabularnewline
	\hline

	%Tipo de CU
	\textbf{Tipo} & \multicolumn{7}{p{194pt}|}{Primario}\tabularnewline
	\hline

	%Requisitos del CU
	\textbf{Referencias} & \multicolumn{2}{p{110pt}|}{} & \multicolumn{5}{p{84pt}|}{}\tabularnewline
	\hline

	%Precondiciones
	\textbf{Precondición} & \multicolumn{7}{p{194pt}|}{Se tiene que haber realizado la prueba en cuestión}\tabularnewline
	\hline

	%Postcondiciones
	\textbf{Postcondición} & \multicolumn{7}{p{194pt}|}{Se adjuntan los datos a la prueba en la base de datos}\tabularnewline
	\hline

	%Autor y fecha
	\textbf{Autor} & Grupo 9  & \multicolumn{2}{p{30pt}|}{
	\textbf{Fecha}} & \today & \multicolumn{2}{p{30pt}|}{
	\textbf{Versión}} & 1.0 \tabularnewline
	\hline
	\end{tabular}

	\vspace{0.5cm}

	%Proposito
	\begin{tabular}{|>{\raggedright}p{337pt}|}
		\hline
		\textbf{Propósito} \tabularnewline \hline
		Almacenar todos los datos relevantes de una prueba.
		\tabularnewline
		\hline
	\end{tabular}

	\vspace{0.5cm}
	%Resumen
	\begin{tabular}{|>{\raggedright}p{337pt}|}
		\hline
		\textbf{Resumen}\tabularnewline
		\hline
		El personal médico encargado, en el apartado correspondiente del sistema,
                añade los datos (resultados, notas, etc.) relevantes para la prueba.
		\tabularnewline
		\hline
	\end{tabular}
	\vspace{0.5cm}
