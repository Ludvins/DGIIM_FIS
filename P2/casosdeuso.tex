

	
		\begin{tabular}{|>{\raggedright}p{58pt}|>{\raggedright}p{109pt}|>{\raggedright}p{1pt}|>{\raggedright}p{17pt}|>{\raggedright}p{28pt}|>{\raggedright}p{0pt}|>{\raggedright}p{18pt}|>{\raggedright}p{20pt}|}
	\hline
	 \textbf{Caso de Uso} &

	%Nombre del CU e identificador
	\multicolumn{5}{p{155pt}|}{Modificacion, creació y borrado de usuarios del sistema}	& \multicolumn{2}{p{39pt}|}{\textbf{25}}\tabularnewline

	\hline

	%Listado de actores
	\textbf{Actores} & \multicolumn{7}{p{194pt}|}{Personal administrativo}\tabularnewline
	\hline

	%Tipo de CU
	\textbf{Tipo} & \multicolumn{7}{p{194pt}|}{Primario y esencial}\tabularnewline
	\hline

	%Requisitos del CU
	\textbf{Referencias} & \multicolumn{2}{p{110pt}|}{Creacion de nuevos usuarios en el sistema y asignacion de permisos a usuarios del sistema} & \multicolumn{5}{p{84pt}|}{}\tabularnewline
	\hline

	%Precondiciones
	\textbf{Precondición} & \multicolumn{7}{p{194pt}|}{Se debe haber realizado un cambio en la situación de un usuario.}\tabularnewline
	\hline

	%Postcondiciones
	\textbf{Postcondición} & \multicolumn{7}{p{194pt}|}{Ante un cambio de un usuario, se deberan establecer los nuevos permisos de este.}\tabularnewline
	\hline

	%Autor y fecha
	\textbf{Autor} & Grupo 9 & \multicolumn{2}{p{30pt}|}{
	\textbf{Fecha}} & 08-04 & \multicolumn{2}{p{30pt}|}{
	\textbf{Versión}} & 1.0 \tabularnewline
	\hline
	\end{tabular}

	\vspace{0.5cm}

	%Proposito
	\begin{tabular}{|>{\raggedright}p{337pt}|}
		\hline
		\textbf{Proposito} \tabularnewline \hline
			Tener un control de los usuarios que se encuentran en el sistema.
		\tabularnewline
		\hline
	\end{tabular}

	\vspace{0.5cm}
	%Resumen
	\begin{tabular}{|>{\raggedright}p{337pt}|}
		\hline
		\textbf{Resumen}\tabularnewline
		\hline
			El sistema debe gestionar la creación o borrado de un usuario cuando sea necesario, asi como ser capaz de modificar un usuario ya existente en el sistema.
		\tabularnewline
		\hline
	\end{tabular}
	\vspace{0.5cm}
	
		\begin{tabular}{|>{\raggedright}p{58pt}|>{\raggedright}p{109pt}|>{\raggedright}p{1pt}|>{\raggedright}p{17pt}|>{\raggedright}p{28pt}|>{\raggedright}p{0pt}|>{\raggedright}p{18pt}|>{\raggedright}p{20pt}|}
	\hline
	 \textbf{Caso de Uso} &

	%Nombre del CU e identificador
	\multicolumn{5}{p{155pt}|}{Emision de facturas a los pacientes}	& \multicolumn{2}{p{39pt}|}{\textbf{26}}\tabularnewline

	\hline

	%Listado de actores
	\textbf{Actores} & \multicolumn{7}{p{194pt}|}{Personal administrativo y pacientes.}\tabularnewline
	\hline

	%Tipo de CU
	\textbf{Tipo} & \multicolumn{7}{p{194pt}|}{Primario y esencial.}\tabularnewline
	\hline

	%Requisitos del CU
	\textbf{Referencias} & \multicolumn{2}{p{110pt}|}{Registro de cobros de las facturas y envio de recordatorio de pago de facturas.} & \multicolumn{5}{p{84pt}|}{}\tabularnewline
	\hline

	%Precondiciones
	\textbf{Precondición} & \multicolumn{7}{p{194pt}|}{Se debe haber realizado el pago por parte del paciente de las facturas pertinentes.}\tabularnewline
	\hline

	%Postcondiciones
	\textbf{Postcondición} & \multicolumn{7}{p{194pt}|}{Registro del pago, no se deben enviar mas recordatorios de pago una vez realizado este.}\tabularnewline
	\hline

	%Autor y fecha
	\textbf{Autor} & Grupo 9 & \multicolumn{2}{p{30pt}|}{
	\textbf{Fecha}} & 08-04 & \multicolumn{2}{p{30pt}|}{
	\textbf{Versión}} & 1.0 \tabularnewline
	\hline
	\end{tabular}

	\vspace{0.5cm}

	%Proposito
	\begin{tabular}{|>{\raggedright}p{337pt}|}
		\hline
		\textbf{Proposito} \tabularnewline \hline
			Administrar el pago de las facturas enviadas a los pacientes.
		\tabularnewline
		\hline
	\end{tabular}

	\vspace{0.5cm}
	%Resumen
	\begin{tabular}{|>{\raggedright}p{337pt}|}
		\hline
		\textbf{Resumen}\tabularnewline
		\hline
			El sistema debe encargarse de administrar y registrar el pago de las facturas de los pacientes, además, se encargara del envio de recordatorios en caso de que el pago no se haya realizado pasada la fecha asignada.
		\tabularnewline
		\hline
	\end{tabular}
	\vspace{0.5cm}
	
	

	\begin{tabular}{|>{\raggedright}p{58pt}|>{\raggedright}p{109pt}|>{\raggedright}p{1pt}|>{\raggedright}p{17pt}|>{\raggedright}p{28pt}|>{\raggedright}p{0pt}|>{\raggedright}p{18pt}|>{\raggedright}p{20pt}|}
	\hline
	 \textbf{Caso de Uso} &

	%Nombre del CU e identificador
	\multicolumn{5}{p{155pt}|}{Pago de facturas a proveedores}	& \multicolumn{2}{p{39pt}|}{\textbf{27}}\tabularnewline

	\hline

	%Listado de actores
	\textbf{Actores} & \multicolumn{7}{p{194pt}|}{Personal administrativo, proveedores y banco}\tabularnewline
	\hline

	%Tipo de CU
	\textbf{Tipo} & \multicolumn{7}{p{194pt}|}{Primario y real}\tabularnewline
	\hline

	%Requisitos del CU
	\textbf{Referencias} & \multicolumn{2}{p{110pt}|}{Registro de pago de facturas} & \multicolumn{5}{p{84pt}|}{}\tabularnewline
	\hline

	%Precondiciones
	\textbf{Precondición} & \multicolumn{7}{p{194pt}|}{Se debe haber realizado un pedido a un proveedor, el pedido debe ser correcto.}\tabularnewline
	\hline

	%Postcondiciones
	\textbf{Postcondición} & \multicolumn{7}{p{194pt}|}{Se debe ajustar el capital tras realizar el pago de la factura.}\tabularnewline
	\hline

	%Autor y fecha
	\textbf{Autor} & Grupo 9 & \multicolumn{2}{p{30pt}|}{
	\textbf{Fecha}} & 08-04 & \multicolumn{2}{p{30pt}|}{
	\textbf{Versión}} & 1.0 \tabularnewline
	\hline
	\end{tabular}

	\vspace{0.5cm}

	%Proposito
	\begin{tabular}{|>{\raggedright}p{337pt}|}
		\hline
		\textbf{Proposito} \tabularnewline \hline
			Realizar el pago de los suministros y productos adquiridos de los proveedores.
		\tabularnewline
		\hline
	\end{tabular}

	\vspace{0.5cm}
	%Resumen
	\begin{tabular}{|>{\raggedright}p{337pt}|}
		\hline
		\textbf{Resumen}\tabularnewline
		\hline
			Se han de pagar las facturas pendientes con los proveedores por su servicio prestado al reabastecer los suministros del centro, además, se debe llevar un registro de estos pagos.
		\tabularnewline
		\hline
	\end{tabular}
	\vspace{0.5cm}

	
	
	
	\begin{tabular}{|>{\raggedright}p{58pt}|>{\raggedright}p{109pt}|>{\raggedright}p{1pt}|>{\raggedright}p{17pt}|>{\raggedright}p{28pt}|>{\raggedright}p{0pt}|>{\raggedright}p{18pt}|>{\raggedright}p{20pt}|}
	\hline
	 \textbf{Caso de Uso} &

	%Nombre del CU e identificador
	\multicolumn{5}{p{155pt}|}{Control de stock, inventario y consumo.}	& \multicolumn{2}{p{39pt}|}{\textbf{28}}\tabularnewline

	\hline

	%Listado de actores
	\textbf{Actores} & \multicolumn{7}{p{194pt}|}{Personal administrativo.}\tabularnewline
	\hline

	%Tipo de CU
	\textbf{Tipo} & \multicolumn{7}{p{194pt}|}{Primario y real}\tabularnewline
	\hline

	%Requisitos del CU
	\textbf{Referencias} & \multicolumn{2}{p{110pt}|}{Registro de los suministros que hay que adquirir} & \multicolumn{5}{p{84pt}|}{}\tabularnewline
	\hline

	%Precondiciones
	\textbf{Precondición} & \multicolumn{7}{p{194pt}|}{Es necesario que se haya producido el uso o consumo de un suministro.}\tabularnewline
	\hline

	%Postcondiciones
	\textbf{Postcondición} & \multicolumn{7}{p{194pt}|}{La necesidad readquirir aquellos suministros utilizados.}\tabularnewline
	\hline

	%Autor y fecha
	\textbf{Autor} & Grupo 9 & \multicolumn{2}{p{30pt}|}{
	\textbf{Fecha}} & 08-04 & \multicolumn{2}{p{30pt}|}{
	\textbf{Versión}} & 1.0 \tabularnewline
	\hline
	\end{tabular}

	\vspace{0.5cm}

	%Proposito
	\begin{tabular}{|>{\raggedright}p{337pt}|}
		\hline
		\textbf{Proposito} \tabularnewline \hline
			Tener el control total de los suministros y el inventario utilizado en el centro.
		\tabularnewline
		\hline
	\end{tabular}

	\vspace{0.5cm}
	%Resumen
	\begin{tabular}{|>{\raggedright}p{337pt}|}
		\hline
		\textbf{Resumen}\tabularnewline
		\hline
			Se debe saber en todo momento y detalle el inventario del centro. Ante el uso de un suministro o la falta de alguno, el sistema debe ser capaz de registrar la necesidad de adquirir mas.
		\tabularnewline
		\hline
	\end{tabular}
	\vspace{0.5cm}

	
	
	\begin{tabular}{|>{\raggedright}p{58pt}|>{\raggedright}p{109pt}|>{\raggedright}p{1pt}|>{\raggedright}p{17pt}|>{\raggedright}p{28pt}|>{\raggedright}p{0pt}|>{\raggedright}p{18pt}|>{\raggedright}p{20pt}|}
	\hline
	 \textbf{Caso de Uso} &

	%Nombre del CU e identificador
	\multicolumn{5}{p{155pt}|}{Asignacion de turnos y guardias}	& \multicolumn{2}{p{39pt}|}{\textbf{29}}\tabularnewline

	\hline

	%Listado de actores
	\textbf{Actores} & \multicolumn{7}{p{194pt}|}{Personal administrativo, personal sanitario, médicos y personal de limpieza.}\tabularnewline
	\hline

	%Tipo de CU
	\textbf{Tipo} & \multicolumn{7}{p{194pt}|}{Primario y esencial}\tabularnewline
	\hline

	%Requisitos del CU
	\textbf{Referencias} & \multicolumn{2}{p{110pt}|}{Asignación de turnos para el personal y asignación de guardias} & \multicolumn{5}{p{84pt}|}{}\tabularnewline
	\hline

	%Precondiciones
	\textbf{Precondición} & \multicolumn{7}{p{194pt}|}{El personal a asignar debe estar disponible en el sistema.}\tabularnewline
	\hline

	%Postcondiciones
	\textbf{Postcondición} & \multicolumn{7}{p{194pt}|}{Una vez asignado un turno o guardia, no se puede asignar otro a la misma hora.}\tabularnewline
	\hline

	%Autor y fecha
	\textbf{Autor} & Grupo 9 & \multicolumn{2}{p{30pt}|}{
	\textbf{Fecha}} & 08-04 & \multicolumn{2}{p{30pt}|}{
	\textbf{Versión}} & 1.0 \tabularnewline
	\hline
	\end{tabular}

	\vspace{0.5cm}

	%Proposito
	\begin{tabular}{|>{\raggedright}p{337pt}|}
		\hline
		\textbf{Proposito} \tabularnewline \hline
			Organizar los horarios de trabajo y de guardias de cada uno de los integrantes del personal.
		\tabularnewline
		\hline
	\end{tabular}

	\vspace{0.5cm}
	%Resumen
	\begin{tabular}{|>{\raggedright}p{337pt}|}
		\hline
		\textbf{Resumen}\tabularnewline
		\hline
			El sistema debe poder organizar los horarios necesarios, tanto regulares como de guardias, teniendo en cuenta que el msimo personal no puede tener dos turnos asignados a la misma hora.
		\tabularnewline
		\hline
	\end{tabular}
	\vspace{0.5cm}

	
	\begin{tabular}{|>{\raggedright}p{58pt}|>{\raggedright}p{109pt}|>{\raggedright}p{1pt}|>{\raggedright}p{17pt}|>{\raggedright}p{28pt}|>{\raggedright}p{0pt}|>{\raggedright}p{18pt}|>{\raggedright}p{20pt}|}
	\hline
	 \textbf{Caso de Uso} &

	%Nombre del CU e identificador
	\multicolumn{5}{p{155pt}|}{Control de las vacaciones del personal}	& \multicolumn{2}{p{39pt}|}{\textbf{30}}\tabularnewline

	\hline

	%Listado de actores
	\textbf{Actores} & \multicolumn{7}{p{194pt}|}{Personal administrativo, personal sanitario, personal de limpieza y médicos.}\tabularnewline
	\hline

	%Tipo de CU
	\textbf{Tipo} & \multicolumn{7}{p{194pt}|}{Primario y esencial}\tabularnewline
	\hline

	%Requisitos del CU
	\textbf{Referencias} & \multicolumn{2}{p{110pt}|}{} & \multicolumn{5}{p{84pt}|}{Asignación de turnos y guardias.}\tabularnewline
	\hline

	%Precondiciones
	\textbf{Precondición} & \multicolumn{7}{p{194pt}|}{Haber trabajado las horas estipuladas necesarias para disponer de dias libres.}\tabularnewline
	\hline

	%Postcondiciones
	\textbf{Postcondición} & \multicolumn{7}{p{194pt}|}{No se deben poder asignar turnos o guardias a aquel personal que se encuentre en periodo vacacional.}\tabularnewline
	\hline

	%Autor y fecha
	\textbf{Autor} & Grupo 9 & \multicolumn{2}{p{30pt}|}{
	\textbf{Fecha}} & 08-04 & \multicolumn{2}{p{30pt}|}{
	\textbf{Versión}} & 1.0 \tabularnewline
	\hline
	\end{tabular}

	\vspace{0.5cm}

	%Proposito
	\begin{tabular}{|>{\raggedright}p{337pt}|}
		\hline
		\textbf{Proposito} \tabularnewline \hline
			Proporcionar al personal de un periodo de descanso de sus funciones.
		\tabularnewline
		\hline
	\end{tabular}

	\vspace{0.5cm}
	%Resumen
	\begin{tabular}{|>{\raggedright}p{337pt}|}
		\hline
		\textbf{Resumen}\tabularnewline
		\hline
			Facilitar a todo integrante del personal un periodo vacacional de descanso.
		\tabularnewline
		\hline
	\end{tabular}
	\vspace{0.5cm}

	
	
	\begin{tabular}{|>{\raggedright}p{58pt}|>{\raggedright}p{109pt}|>{\raggedright}p{1pt}|>{\raggedright}p{17pt}|>{\raggedright}p{28pt}|>{\raggedright}p{0pt}|>{\raggedright}p{18pt}|>{\raggedright}p{20pt}|}
	\hline
	 \textbf{Caso de Uso} &

	%Nombre del CU e identificador
	\multicolumn{5}{p{155pt}|}{Pago al personal}	& \multicolumn{2}{p{39pt}|}{\textbf{31}}\tabularnewline

	\hline

	%Listado de actores
	\textbf{Actores} & \multicolumn{7}{p{194pt}|}{Banco, personal administrativo, personal de limpieza, personal sanitario fs y médicos.}\tabularnewline
	\hline

	%Tipo de CU
	\textbf{Tipo} & \multicolumn{7}{p{194pt}|}{Primario y esencial}\tabularnewline
	\hline

	%Requisitos del CU
	\textbf{Referencias} & \multicolumn{2}{p{110pt}|}{Registro de pago al personal} & \multicolumn{5}{p{84pt}|}{}\tabularnewline
	\hline

	%Precondiciones
	\textbf{Precondición} & \multicolumn{7}{p{194pt}|}{El personal correspondiente ha realizado correctamente el trabajo, y no se ha producido ningún incidente .}\tabularnewline
	\hline

	%Postcondiciones
	\textbf{Postcondición} & \multicolumn{7}{p{194pt}|}{Todo el personal ha recibido el salario correspondiente.}\tabularnewline
	\hline

	%Autor y fecha
	\textbf{Autor} &  Grupo 9 & \multicolumn{2}{p{30pt}|}{
	\textbf{Fecha}} & 08-04 & \multicolumn{2}{p{30pt}|}{
	\textbf{Versión}} & 1.0 \tabularnewline
	\hline
	\end{tabular}

	\vspace{0.5cm}

	%Proposito
	\begin{tabular}{|>{\raggedright}p{337pt}|}
		\hline
		\textbf{Proposito} \tabularnewline \hline
	Realizar el pago correspondiente al indicado en su contrato a cada integrante del personal.
		\tabularnewline
		\hline
	\end{tabular}

	\vspace{0.5cm}
	%Resumen
	\begin{tabular}{|>{\raggedright}p{337pt}|}
		\hline
		\textbf{Resumen}\tabularnewline
		\hline
		El sistema debe ser capaz de realizar el pago a cada uno de los integrantes del personal, teniendo en cuenta los diferentes salarios de cada uno.
		Además debe dejar registro del pago realizado.
		\tabularnewline
		\hline
	\end{tabular}
	\vspace{0.5cm}
	