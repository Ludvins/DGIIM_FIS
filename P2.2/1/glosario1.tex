\begin{itemize}
\item Cita : Momento en el que un médico y un paciente acuerdan encontrarse para tratar algún tema.
\item Molestia : Dolencia que sufre el paciente y le lleva a concertar una cita con el médico para estudiar si hay algún problema en su organismo.
\item HC-Historial Clínico: Es una hoja en la que se exponen tanto los datos del paciente como las enfermedades que ha sufrido el paciente en el pasado.
\item Cura : Hablaremos de cura como el momento en el que se encuentra la enfermedad que se ha dado en el paciente
\item Exploración: Proceso en el que se investigan los posibles problemas en el cuerpo del paciente para buscar la cura.
\item Recetar : Acto de indicar a un paciente qué medicamento debe tomarse y escribirlo en una hoja para que el paciente lo tenga y la persona encargada en la farmacia pueda proporcionarle el medicamento
\item Tratamiento : Conjunto de medicinas o cuidados que deben darse a un paciente que lo necesite

\item Prueba médica : Es una prueba realizada por máquinas especializadas encargadas de buscar por medio de la tecnología los posibles problemas que pueda tener el paciente

\item Médico especialista : Médico que se encarga de una zona específica del cuerpo de la que tiene más conocimientos que otro médico no especialista en la zona.

\item Consulta : Sinónimo de Cita.
\item Modificación puntual : Modificación que se hace debido a una urgencia, que no se hace de forma sistemática si no en un caso extraordinario.
\item Permiso sobre los datos: Cada tipo de usuario en el sistema posee diferentes permisos de acceso sobre los datos del sistema. Los pacientes no podrán acceder a los datos de otros pacientes, sino solo a los suyos. Los médicos sí podrán acceder a los de todos los pacientes. El personal administrativo tampoco tendrá acceso a los datos médicos de los pacientes.

\item Alta de paciente: Se produce cuando un paciente está capacitado para salir del hospital después de una estancia en el mismo. Es el acto de comunicarle al paciente que puede marcharse del hospital sin peligro ,en principio, de recaer en su enfermedad.

  \item Contratación : Acto entre una persona ajena al centro y otra perteneciente al hospital, en el que la persona ajena firma un contrato en el que se acuerda que comenzará a trabajar en el hospital y se fija su horario, salario, lugar de trabajo, etc.

  \end{itemize}
