

\begin{tabular}{|>{\raggedright}p{58pt}|>{\raggedright}p{109pt}|>{\raggedright}p{1pt}|>{\raggedright}p{17pt}|>{\raggedright}p{28pt}|>{\raggedright}p{0pt}|>{\raggedright}p{18pt}|>{\raggedright}p{20pt}|}
	\hline
	 \textbf{Caso de Uso} &

	%Nombre del CU e identificador
	\multicolumn{5}{p{155pt}|}{Iniciar consulta}	& \multicolumn{2}{p{39pt}|}{\textbf{CU1}}\tabularnewline

	\hline

	%Listado de actores
	\textbf{Actores} & \multicolumn{7}{p{194pt}|}{Paciente, Sanitario}\tabularnewline
	\hline

	%Tipo de CU
	\textbf{Tipo} & \multicolumn{7}{p{194pt}|}{Principal,real}\tabularnewline
	\hline

	%Requisitos del CU
	\textbf{Referencias} & \multicolumn{2}{p{110pt}|}{El paciente deberá sentir algun tipo de molestia} & \multicolumn{5}{p{84pt}|}{Crear HC, Abrir HC}\tabularnewline
	\hline

	%Precondiciones
	\textbf{Precondición} & \multicolumn{7}{p{194pt}|}{El paciente debe haber solicitado una cita y se le debe haber concedido}\tabularnewline
	\hline

	%Postcondiciones
	\textbf{Postcondición} & \multicolumn{7}{p{194pt}|}{La cita entre el paciente y el sanitario ha comenzado}\tabularnewline
	\hline

	%Autor y fecha
	\textbf{Autor} & Javier Sáez & \multicolumn{2}{p{30pt}|}{
	\textbf{Fecha}} & \today & \multicolumn{2}{p{30pt}|}{
	\textbf{Versión}} & 1.0 \tabularnewline
	\hline
	\end{tabular}

	\vspace{0.5cm}

	%Proposito
	\begin{tabular}{|>{\raggedright}p{337pt}|}
		\hline
		\textbf{Propósito} \tabularnewline \hline
			Empezar la cita para tratar de hallar la enfermedad del paciente
		\tabularnewline
		\hline
	\end{tabular}

	\vspace{0.5cm}
	%Resumen
	\begin{tabular}{|>{\raggedright}p{337pt}|}
		\hline
		\textbf{Resumen}\tabularnewline
		\hline
			Tras la solicitud y concesión de la cita, el paciente llegará a la consulta y entrará a la sala donde se le indique para comenzar la conversación con el médico.
		\tabularnewline
		\hline
	\end{tabular}
	\vspace{0.5cm}

	%Plantilla Extendida

	\begin{tabular}{|>{\raggedright}p{11pt}|>{\raggedright}p{138pt}|>{\raggedright}p{10pt}|>{\raggedright}p{140pt}|}
		\hline
		\multicolumn{4}{|p{301pt}|}{
		\textbf{Curso normal (básico)}}\tabularnewline
		\hline
		\centering 1 & Paciente: Llega al hospital & \centering  2 & Indica la sala al paciente\tabularnewline
		\hline
		\centering 3 & Paciente: Entra en la sala marcada por el sistema & \centering  &  \tabularnewline
		\hline
		\centering 4& Médico: Comienza la conversación con el paciente & \centering 5 & El sistema marca la cita como comenzada\\ Include CU3- Abrir HC\\
\textbf{Punto de extensión:} Crear HC\tabularnewline
		\hline

	\end{tabular}

	\vspace{0.5cm}
	\newpage

	%CU Alternos
	\begin{tabular}{|>{\raggedright}p{11pt}|>{\raggedright}p{56pt}|>{\raggedright}p{91pt}|>{\raggedright}p{46pt}|>{\raggedright}p{83pt}|}
		\hline
		\multicolumn{5}{|p{337pt}|}{\textbf{Cursos alternos}}\tabularnewline
		\hline
		\centering \textbf{1a} & \multicolumn{4}{p{278pt}|}{
		 El médico no está en el hospital}\tabularnewline
		\hline
		\centering \textbf{1b} & \multicolumn{4}{p{278pt}|}{ El sistema indica al paciente que su cita es pospuesta e intenta reasignar una cita
		}\tabularnewline
		\hline
		\centering \textbf{3b} & \multicolumn{4}{p{278pt}|}{El médico no ha terminado su cita anterior.}\tabularnewline
		\hline
		\hline
		\centering \textbf{3c} & \multicolumn{4}{p{278pt}|}{El sistema pospone la hora de la cita hasta que termine la cita actual y el paciente debe esperar.}\tabularnewline
		\hline
	\end{tabular}
	\vspace{0.5cm}

	%Otros Datos
	\begin{tabular}{|>{\raggedright}p{11pt}|>{\raggedright}p{56pt}|>{\raggedright}p{88pt}|>{\raggedright}p{50pt}|>{\raggedright}p{83pt}|}
		\hline
		\multicolumn{5}{|p{337pt}|}{\textbf{Otros datos}}\tabularnewline
		\hline
		
		 \multicolumn{2}{|p{68pt}|}{
		\textbf{Frecuencia esperada}} & Alta \quad & \textbf{Rendimiento} & 
		Alto \tabularnewline
		\hline
		
		
		 \multicolumn{2}{|p{68pt}|}{
		\textbf{Importancia}} & Vital\quad  & \textbf{Urgencia} &
		Media \tabularnewline
		\hline
		\multicolumn{2}{|p{68pt}|}{\textbf{Estado}} &  \quad  & \textbf{Estabilidad} &
		Alta \tabularnewline
		\hline
	\end{tabular}


	%Comentarios
	\vspace{0.5cm}
	\begin{tabular}{|>{\raggedright}p{337pt}|}
		\hline
		\multicolumn{1}{|p{337pt}|}{\textbf{Comentarios}}\tabularnewline
		\hline
		\multicolumn{1}{|p{337pt}|}{Si al abrir el historial clínico del paciente este no existe, se lanzará un mensaje de error y saltará la ventana de crear un nuevo HC.} \tabularnewline
		\hline
	\end{tabular}
	
	%Caso de extensión
	
	\begin{tabular}{|>{\raggedright}p{11pt}|>{\raggedright}p{138pt}|>{\raggedright}p{10pt}|>{\raggedright}p{140pt}|}
		\hline
		\multicolumn{4}{|p{301pt}|}{
		\textbf{Caso de extensión: Crear Historial Clínico -\textbf{CU2}}}\tabularnewline
		\hline
		\centering 1 & Médico : Intenta abrir el HC del paciente & \centering  2 & Indica que no existe ese historial clínico\tabularnewline
		\hline
		\centering 3 & Médico: Solicita los datos al paciente y los introduce en el sistema & \centering  4& Guarda los datos del paciente y crea en la base de datos una tabla para él con ellos y un historial clínico vacío. \tabularnewline
		\hline

	\end{tabular}
	\vspace{0.5cm}


