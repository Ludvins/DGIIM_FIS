

\begin{tabular}{|>{\raggedright}p{58pt}|>{\raggedright}p{109pt}|>{\raggedright}p{1pt}|>{\raggedright}p{17pt}|>{\raggedright}p{28pt}|>{\raggedright}p{0pt}|>{\raggedright}p{18pt}|>{\raggedright}p{20pt}|}
	\hline
	 \textbf{Caso de Uso} &

	%Nombre del CU e identificador
	\multicolumn{5}{p{155pt}|}{Cura}	& \multicolumn{2}{p{39pt}|}{\textbf{CU6}}\tabularnewline

	\hline

	%Listado de actores
	\textbf{Actores} & \multicolumn{7}{p{194pt}|}{Sanitario}\tabularnewline
	\hline

	%Tipo de CU
	\textbf{Tipo} & \multicolumn{7}{p{194pt}|}{Primario, Real}\tabularnewline
	\hline

	%Requisitos del CU
	\textbf{Referencias} & \multicolumn{2}{p{110pt}|}{El médico puede necesitar tiempo para realizar su tarea} & \multicolumn{5}{p{84pt}|}{}\tabularnewline
	\hline

	%Precondiciones
	\textbf{Precondición} & \multicolumn{7}{p{194pt}|}{El paciente estará marcado como explorado y la exploración debe haber terminado}\tabularnewline
	\hline

	%Postcondiciones
	\textbf{Postcondición} & \multicolumn{7}{p{194pt}|}{Se habrá determinado la enfermedad del cliente}\tabularnewline
	\hline

	%Autor y fecha
	\textbf{Autor} & Javier Sáez & \multicolumn{2}{p{30pt}|}{
	\textbf{Fecha}} & \today & \multicolumn{2}{p{30pt}|}{
	\textbf{Versión}} & 1.0 \tabularnewline
	\hline
	\end{tabular}

	\vspace{0.5cm}

	%Proposito
	\begin{tabular}{|>{\raggedright}p{337pt}|}
		\hline
		\textbf{Propósito} \tabularnewline \hline
			Determinar de forma precisa la enfermedad del paciente
		\tabularnewline
		\hline
	\end{tabular}

	\vspace{0.5cm}
	%Resumen
	\begin{tabular}{|>{\raggedright}p{337pt}|}
		\hline
		\textbf{Resumen}\tabularnewline
		\hline
			El médico, tras haber analizado al paciente, deberá viendo los síntomas y dónde está el problema, determinar la enfermedad e introducirla en el sistema
		\tabularnewline
		\hline
	\end{tabular}
	\vspace{0.5cm}

	%Plantilla Extendida

	\begin{tabular}{|>{\raggedright}p{11pt}|>{\raggedright}p{138pt}|>{\raggedright}p{10pt}|>{\raggedright}p{140pt}|}
		\hline
		\multicolumn{4}{|p{301pt}|}{
		\textbf{Curso normal (básico)}}\tabularnewline
		\hline
		\centering 1 & Sanitario: Termina la exploración & \centering  & \tabularnewline
		\hline
		\centering 2 & Sanitario: Determina la enfermedad y la introduce en el sistema & \centering 3 &  Guarda en el sistema la enfermedad que padece el paciente\tabularnewline
		\hline
	\end{tabular}

	\vspace{0.5cm}
	\newpage

	%CU Alternos
	\begin{tabular}{|>{\raggedright}p{11pt}|>{\raggedright}p{56pt}|>{\raggedright}p{91pt}|>{\raggedright}p{46pt}|>{\raggedright}p{83pt}|}
		\hline
		\multicolumn{5}{|p{337pt}|}{\textbf{Cursos alternos}}\tabularnewline
		\hline
		\centering \textbf{2a} & \multicolumn{4}{p{278pt}|}{
		 El médico no es capaz de determinar la enfermedad}\tabularnewline
		\hline
		\centering \textbf{2b} & \multicolumn{4}{p{278pt}|}{ El médico consulta en su base de datos qué tipo de enfermerdad puede ser, y se vuelve al paso 2.
		}\tabularnewline
		\hline
		\centering  3a& \multicolumn{4}{p{278pt}|}{Si aun así el médico sigue sin poder ser capaz de determinar la enfermedad, se pasará al \textbf{CU10}, pasar a un especialista }\tabularnewline
		\hline
		\centering  3b& \multicolumn{4}{p{278pt}|}{El sistema anota en su base de datos lo ocurrido, e intentará concertar una cita con el especialista. }\tabularnewline
		\hline
	\end{tabular}
	\vspace{0.5cm}

	%Otros Datos
	\begin{tabular}{|>{\raggedright}p{11pt}|>{\raggedright}p{56pt}|>{\raggedright}p{88pt}|>{\raggedright}p{50pt}|>{\raggedright}p{83pt}|}
		\hline
		\multicolumn{5}{|p{337pt}|}{\textbf{Otros datos}}\tabularnewline
		\hline
		
		 \multicolumn{2}{|p{68pt}|}{
		\textbf{Frecuencia esperada}} & Alta \quad & \textbf{Rendimiento} & 
		Alta \tabularnewline
		\hline
		
		
		 \multicolumn{2}{|p{68pt}|}{
		\textbf{Importancia}} & Alta\quad  & \textbf{Urgencia} &
		Moderada \tabularnewline
		\hline
		\multicolumn{2}{|p{68pt}|}{\textbf{Estado}} & En espera \quad  & \textbf{Estabilidad} &
		Alta \tabularnewline
		\hline
	\end{tabular}

	



