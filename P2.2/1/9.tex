

\begin{tabular}{|>{\raggedright}p{58pt}|>{\raggedright}p{109pt}|>{\raggedright}p{1pt}|>{\raggedright}p{17pt}|>{\raggedright}p{28pt}|>{\raggedright}p{0pt}|>{\raggedright}p{18pt}|>{\raggedright}p{20pt}|}
	\hline
	 \textbf{Caso de Uso} &

	%Nombre del CU e identificador
	\multicolumn{5}{p{155pt}|}{Pedir prueba médica}	& \multicolumn{2}{p{39pt}|}{\textbf{CU9}}\tabularnewline

	\hline

	%Listado de actores
	\textbf{Actores} & \multicolumn{7}{p{194pt}|}{Médico, Paciente}\tabularnewline
	\hline

	%Tipo de CU
	\textbf{Tipo} & \multicolumn{7}{p{194pt}|}{Secundario, Esencial}\tabularnewline
	\hline

	%Requisitos del CU
	\textbf{Referencias} & \multicolumn{2}{p{110pt}|}{Debe ser necesaria una prueba extra para el paciente} & \multicolumn{5}{p{84pt}|}{\textbf{CU6}- Cura}\tabularnewline
	\hline

	%Precondiciones
	\textbf{Precondición} & \multicolumn{7}{p{194pt}|}{El médico debe haber indicado que no es capaz de reconocer la enfermedad del paciente}\tabularnewline
	\hline

	%Postcondiciones
	\textbf{Postcondición} & \multicolumn{7}{p{194pt}|}{El paciente tendrá una cita para una prueba médica asignada para determinar su enfermedad}\tabularnewline
	\hline

	%Autor y fecha
	\textbf{Autor} & Javier Sáez  & \multicolumn{2}{p{30pt}|}{
	\textbf{Fecha}} & \today & \multicolumn{2}{p{30pt}|}{
	\textbf{Versión}} & 1.0 \tabularnewline
	\hline
	\end{tabular}

	\vspace{0.5cm}

	%Proposito
	\begin{tabular}{|>{\raggedright}p{337pt}|}
		\hline
		\textbf{Propósito} \tabularnewline \hline
			Dar una cita para una prueba médica a un paciente
		\tabularnewline
		\hline
	\end{tabular}

	\vspace{0.5cm}
	%Resumen
	\begin{tabular}{|>{\raggedright}p{337pt}|}
		\hline
		\textbf{Resumen}\tabularnewline
		\hline
			El médico no ha sido capaz de determinar la enfermedad con la exploración simple. Es por ello que solicita al sistema que asigne al paciente una cita para una prueba médica para poder determinar con seguridad la enfermedad que padece.
		\tabularnewline
		\hline
	\end{tabular}
	\vspace{0.5cm}

	%Plantilla Extendida

	\begin{tabular}{|>{\raggedright}p{11pt}|>{\raggedright}p{138pt}|>{\raggedright}p{10pt}|>{\raggedright}p{140pt}|}
		\hline
		\multicolumn{4}{|p{301pt}|}{
		\textbf{Curso normal (básico)}}\tabularnewline
		\hline
		\centering 1 & Médico: No identifica la enfermedad. Pide al sistema que busque una cita para una prueba médica & \centering 2 & Busca una cita para una prueba para el paciente \tabularnewline
		\hline
		\centering  & & \centering 3 &  Indica al médico la cita que podrá tener el paciente\tabularnewline
		\hline
		\centering  4& El médico comunica al paciente cuándo es su cita para determinar su enfermedad  & \centering  & \tabularnewline
		\hline

	\end{tabular}

	\vspace{0.5cm}




	%Otros Datos
	\begin{tabular}{|>{\raggedright}p{11pt}|>{\raggedright}p{56pt}|>{\raggedright}p{88pt}|>{\raggedright}p{50pt}|>{\raggedright}p{83pt}|}
		\hline
		\multicolumn{5}{|p{337pt}|}{\textbf{Otros datos}}\tabularnewline
		\hline

		 \multicolumn{2}{|p{68pt}|}{
		\textbf{Frecuencia esperada}} & Baja \quad & \textbf{Rendimiento} &
		Alto \tabularnewline
		\hline


		 \multicolumn{2}{|p{68pt}|}{
		\textbf{Importancia}} & Alta \quad  & \textbf{Urgencia} &
		Moderada \tabularnewline
		\hline
		\multicolumn{2}{|p{68pt}|}{\textbf{Estado}} & Activo \quad  & \textbf{Estabilidad} &
		Alta\tabularnewline
		\hline
	\end{tabular}

	%Comentarios
	\vspace{0.5cm}
	\begin{tabular}{|>{\raggedright}p{337pt}|}
		\hline
		\multicolumn{1}{|p{337pt}|}{\textbf{Comentarios}}\tabularnewline
		\hline
		\multicolumn{1}{|p{337pt}|}{No habrá cursos del CU alternativos. L importancia puede también ser Moderada si no hay prisa en determinar la enfermedad.} \tabularnewline
		\hline
	\end{tabular}
