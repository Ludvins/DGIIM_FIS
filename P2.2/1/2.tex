

\begin{tabular}{|>{\raggedright}p{58pt}|>{\raggedright}p{109pt}|>{\raggedright}p{1pt}|>{\raggedright}p{17pt}|>{\raggedright}p{28pt}|>{\raggedright}p{0pt}|>{\raggedright}p{18pt}|>{\raggedright}p{20pt}|}
	\hline
	 \textbf{Caso de Uso} &

	%Nombre del CU e identificador
	\multicolumn{5}{p{155pt}|}{Crear HC}	& \multicolumn{2}{p{39pt}|}{\textbf{CU2}}\tabularnewline

	\hline

	%Listado de actores
	\textbf{Actores} & \multicolumn{7}{p{194pt}|}{Paciente, Sanitario}\tabularnewline
	\hline

	%Tipo de CU
	\textbf{Tipo} & \multicolumn{7}{p{194pt}|}{Esencial}\tabularnewline
	\hline

	%Requisitos del CU
	\textbf{Referencias} & \multicolumn{2}{p{110pt}|}{Indicamos que requisitos se pueden incluir dentro} & \multicolumn{5}{p{84pt}|}{Iniciar consulta}\tabularnewline
	\hline

	%Precondiciones
	\textbf{Precondición} & \multicolumn{7}{p{194pt}|}{Debe estar la cita comenzada y el cliente no tener HC creado}\tabularnewline
	\hline

	%Postcondiciones
	\textbf{Postcondición} & \multicolumn{7}{p{194pt}|}{Habrá una ficha de Historial Clínico para el paciente}\tabularnewline
	\hline

	%Autor y fecha
	\textbf{Autor} & Javier Sáez  & \multicolumn{2}{p{30pt}|}{
	\textbf{Fecha}} & \today & \multicolumn{2}{p{30pt}|}{
	\textbf{Versión}} & 1.0 \tabularnewline
	\hline
	\end{tabular}

	\vspace{0.5cm}

	%Proposito
	\begin{tabular}{|>{\raggedright}p{337pt}|}
		\hline
		\textbf{Propósito} \tabularnewline \hline
			Crear un historial clínico para el paciente
		\tabularnewline
		\hline
	\end{tabular}

	\vspace{0.5cm}
	%Resumen
	\begin{tabular}{|>{\raggedright}p{337pt}|}
		\hline
		\textbf{Resumen}\tabularnewline
		\hline
			Cuando el sanitario va a consultar el historial clínico del paciente y no lo tiene, el sistema lo crea y asigna los campos vacíos
		\tabularnewline
		\hline
	\end{tabular}
	\vspace{0.5cm}

	%Plantilla Extendida

	\begin{tabular}{|>{\raggedright}p{11pt}|>{\raggedright}p{138pt}|>{\raggedright}p{10pt}|>{\raggedright}p{140pt}|}
		\hline
		\multicolumn{4}{|p{301pt}|}{
		\textbf{Curso normal (básico)}}\tabularnewline
		\hline
		\centering 1 & Sanitario: Encuentra que no hay HC del cliente y dice al sistema que creee un HC nuevo & \centering 2 & Abre el formulario de crear un HC\tabularnewline
		\hline
		\centering 3 & Sanitario: Pide los datos al paciente & \centering  &  \tabularnewline
		\hline
		\centering 4 & Paciente: Da los datos al médico & \centering  & \tabularnewline
		\hline
		\centering 5 & Médico: Introduce los datos en el sistema & \centering 6 & El sistema crea un HC para el paciente con los datos dados.\tabularnewline
		\hline

	\end{tabular}

	\vspace{0.5cm}

	%CU Alternos
	\begin{tabular}{|>{\raggedright}p{11pt}|>{\raggedright}p{56pt}|>{\raggedright}p{91pt}|>{\raggedright}p{46pt}|>{\raggedright}p{83pt}|}
		\hline
		\multicolumn{5}{|p{337pt}|}{\textbf{Cursos alternos}}\tabularnewline
		\hline
		\centering \textbf{6a} & \multicolumn{4}{p{278pt}|}{
		 Algún dato falta o es erróneo}\tabularnewline
		\hline
		\centering \textbf{6b} & \multicolumn{4}{p{278pt}|}{El sistema pide de nuevo el dato al médico
		}\tabularnewline
		\hline
		\centering \textbf{6c} & \multicolumn{4}{p{278pt}|}{El médico pide el dato al paciente de nuevo y lo introduce
		}\tabularnewline
		\hline
		\centering \textbf{6d} & \multicolumn{4}{p{278pt}|}{El médico vuelve a introducir el dato en el sistema
		}\tabularnewline
		\hline
		\centering \textbf{6e} & \multicolumn{4}{p{278pt}|}{El sistema crea finalmente el historial clínico.
		}\tabularnewline
		\hline
	\end{tabular}
	\vspace{0.5cm}

	%Otros Datos
	\begin{tabular}{|>{\raggedright}p{11pt}|>{\raggedright}p{56pt}|>{\raggedright}p{88pt}|>{\raggedright}p{50pt}|>{\raggedright}p{83pt}|}
		\hline
		\multicolumn{5}{|p{337pt}|}{\textbf{Otros datos}}\tabularnewline
		\hline

		 \multicolumn{2}{|p{68pt}|}{
		\textbf{Frecuencia esperada}} & Baja \quad & \textbf{Rendimiento} &
		Alto \tabularnewline
		\hline


		 \multicolumn{2}{|p{68pt}|}{
		\textbf{Importancia}} & Moderada\quad  & \textbf{Urgencia} &
		Baja\tabularnewline
		\hline
		\multicolumn{2}{|p{68pt}|}{\textbf{Estado}} &  \quad  & \textbf{Estabilidad} &
		Alta \tabularnewline
		\hline
	\end{tabular}

	%Comentarios
	\vspace{0.5cm}
	\begin{tabular}{|>{\raggedright}p{337pt}|}
		\hline
		\multicolumn{1}{|p{337pt}|}{\textbf{Comentarios}}\tabularnewline
		\hline
		\multicolumn{1}{|p{337pt}|}{Será un caso de uso muy poco habitual} \tabularnewline
		\hline
	\end{tabular}
