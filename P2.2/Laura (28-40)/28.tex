\begin{tabular}{|>{\raggedright}p{58pt}|>{\raggedright}p{109pt}|>{\raggedright}p{1pt}|>{\raggedright}p{17pt}|>{\raggedright}p{28pt}|>{\raggedright}p{0pt}|>{\raggedright}p{18pt}|>{\raggedright}p{20pt}|}
	\hline
	 \textbf{Caso de Uso} &

	%Nombre del CU e identificador
	\multicolumn{5}{p{155pt}|}{Liberar Sanitario de Consulta Médica
}	& \multicolumn{2}{p{39pt}|}{\textbf{Identificador} CU-28}\tabularnewline

	\hline

	%Listado de actores
	\textbf{Actores} & \multicolumn{7}{p{194pt}|}{Sanitario (I), Administrativo}\tabularnewline
	\hline

	%Tipo de CU
	\textbf{Tipo} & \multicolumn{7}{p{194pt}|}{Secundario \textbar Real}\tabularnewline
	\hline

	%Requisitos del CU
	\textbf{Referencias} & \multicolumn{2}{p{110pt}|}{Que no se quede ninguna consulta sin sanitario asignado.} & \multicolumn{5}{p{84pt}|}{Asignar sanitario a consulta médica.}\tabularnewline
	\hline

	%Precondiciones
	\textbf{Precondición} & \multicolumn{7}{p{194pt}|}{Debe de estar asignados al menos dos sanitarios a la consulta. Así, siempre habrá alguien encargado de dicha consulta.}\tabularnewline
	\hline

	%Postcondiciones
	\textbf{Postcondición} & \multicolumn{7}{p{194pt}|}{El sanitario ha sido liberado de la consulta, no siendo esta ya una obligación suya.}\tabularnewline
	\hline

	%Autor y fecha
	\textbf{Autor} & Laura Gómez Garrido  & \multicolumn{2}{p{30pt}|}{
	\textbf{Fecha}} & \today & \multicolumn{2}{p{30pt}|}{
	\textbf{Versión}} & 1.0 \tabularnewline
	\hline
	\end{tabular}

	\vspace{0.5cm}

	%Proposito
	\begin{tabular}{|>{\raggedright}p{337pt}|}
		\hline
		\textbf{Propósito} \tabularnewline \hline
			Junto a (CU-29, Asignar sanitario a consulta médica), modificarán las distintas consultas para poder subsanar errores o cambios.
		\tabularnewline
		\hline
	\end{tabular}

	\vspace{0.5cm}
	%Resumen
	\begin{tabular}{|>{\raggedright}p{337pt}|}
		\hline
		\textbf{Resumen}\tabularnewline
		\hline
A veces es necesario modificar el personal asignado a una determinada consulta médica, ya sea por cambios en los turnos, solicitudes del personal o del paciente o cualquier otra causa justificada.
		\tabularnewline
		\hline
	\end{tabular}
	\vspace{0.5cm}

	%Plantilla Extendida

	\begin{tabular}{|>{\raggedright}p{11pt}|>{\raggedright}p{138pt}|>{\raggedright}p{10pt}|>{\raggedright}p{140pt}|}
		\hline
		\multicolumn{4}{|p{301pt}|}{
		\textbf{Curso normal (básico)}}\tabularnewline
		\hline
		\centering 1 & Sanitario: Solicita su liberación de una determinada consulta médica. & 2 \centering  & Notifica al administrativo de la existencia una solicitud. \tabularnewline
		\hline
		\centering 3 & Admnistrativo: Comprueba la viabilidad la solicitud. En caso de ser necesario que otro sanitario se ocupe de dicha consulta, busca a otro encargado. & \centering  &  \tabularnewline
		\hline
		\centering 4 & Administrativo: Aprueba o rechaza la solicitud a través del sistema. & \centering 5 & Notificar de la aprobación o rechazo de su solicitud al Sanitario\tabularnewline
		\hline
		\centering 6 & Sanitario: Comprobar el estado de su solicitud. & \centering  & \tabularnewline
		\hline
	\end{tabular}

	\vspace{0.5cm}
	\newpage

	%CU Alternos
	\begin{tabular}{|>{\raggedright}p{11pt}|>{\raggedright}p{56pt}|>{\raggedright}p{91pt}|>{\raggedright}p{46pt}|>{\raggedright}p{83pt}|}
		\hline
		\multicolumn{5}{|p{337pt}|}{\textbf{Cursos alternos}}\tabularnewline
		\hline
		\centering \textbf{1a} & \multicolumn{4}{p{300pt}|}{ \textbf{1a $\rightarrow$ 3}
		Es el propio administrativo quien realiza la liberación sin necesidad de una solicitud. Empezamos por el paso 3 del curso normal. }\tabularnewline
		\hline
		\centering \textbf{3a} & \multicolumn{4}{p{300pt}|}{ \textbf{3 $\rightarrow$ 3a $\rightarrow$ 3b}
		Es necesario encontrar otro administrativo que se encargue de la consulta. Se inicia un diálogo personal entre el administrativo y los distintos sanitarios para ver si alguno está dispuesto a aceptar encargarse de dicha consulta.}\tabularnewline
		\hline
		\centering  \textbf{3b} & \multicolumn{4}{p{300pt}|}{ \textbf{3a $\rightarrow$ 3b $\rightarrow$ 4 }
		Si encuentra alguno, ver (CU-29, Asignar sanitario a consulta médica).}\tabularnewline
		\hline
	\end{tabular}
	\vspace{0.5cm}

	%Otros Datos
	\begin{tabular}{|>{\raggedright}p{11pt}|>{\raggedright}p{56pt}|>{\raggedright}p{88pt}|>{\raggedright}p{50pt}|>{\raggedright}p{83pt}|}
		\hline
		\multicolumn{5}{|p{337pt}|}{\textbf{Otros datos}}\tabularnewline
		\hline

		 \multicolumn{2}{|p{68pt}|}{
		\textbf{Frecuencia \newline esperada}} & Decenas de veces al año. \quad & \textbf{Rendimiento} &
		Medio \tabularnewline
		\hline


		 \multicolumn{2}{|p{68pt}|}{
		\textbf{Importancia}} & Moderada \quad  & \textbf{Urgencia} &
Moderada \tabularnewline
		\hline
		\multicolumn{2}{|p{68pt}|}{\textbf{Estado}} & Activo \quad  & \textbf{Estabilidad} &
		Alta \tabularnewline
		\hline
	\end{tabular}

	%Comentarios
	\vspace{0.5cm}
	\begin{tabular}{|>{\raggedright}p{337pt}|}
		\hline
		\multicolumn{1}{|p{337pt}|}{\textbf{Comentarios}}\tabularnewline
		\hline
		\multicolumn{1}{|p{337pt}|}{} \tabularnewline
		\hline
	\end{tabular}
