\begin{tabular}{|>{\raggedright}p{58pt}|>{\raggedright}p{109pt}|>{\raggedright}p{1pt}|>{\raggedright}p{17pt}|>{\raggedright}p{28pt}|>{\raggedright}p{0pt}|>{\raggedright}p{18pt}|>{\raggedright}p{20pt}|}
	\hline
	 \textbf{Caso de Uso} &

	%Nombre del CU e identificador
	\multicolumn{5}{p{155pt}|}{Alta de Sanitario
}	& \multicolumn{2}{p{39pt}|}{\textbf{Identificador} CU-30}\tabularnewline

	\hline

	%Listado de actores
	\textbf{Actores} & \multicolumn{7}{p{194pt}|}{Sanitario, Compañía Médica(I)}\tabularnewline
	\hline

	%Tipo de CU
	\textbf{Tipo} & \multicolumn{7}{p{194pt}|}{Primario \textbar Esencial}\tabularnewline
	\hline

	%Requisitos del CU
	\textbf{Referencias} & \multicolumn{2}{p{110pt}|}{Debe de asegurarse de que el sanitario es un personal cualificado para su puesto.} & \multicolumn{5}{p{84pt}|}{Contratación, Baja de Sanitario, Fin del Contrato}\tabularnewline
	\hline

	%Precondiciones
	\textbf{Precondición} & \multicolumn{7}{p{194pt}|}{El sanitario debe de haber sido contratado.}\tabularnewline
	\hline

	%Postcondiciones
	\textbf{Postcondición} & \multicolumn{7}{p{194pt}|}{El sanitario puede ejercer como tal dentro de nuestro hospital.}\tabularnewline
	\hline

	%Autor y fecha
	\textbf{Autor} & Laura Gómez Garrido  & \multicolumn{2}{p{30pt}|}{
	\textbf{Fecha}} & \today & \multicolumn{2}{p{30pt}|}{
	\textbf{Versión}} & 1.0 \tabularnewline
	\hline
	\end{tabular}

	\vspace{0.5cm}

	%Proposito
	\begin{tabular}{|>{\raggedright}p{337pt}|}
		\hline
		\textbf{Propósito} \tabularnewline \hline
			Junto a (CU-25, Contratación), (CU-27, Fin de Contrato) y (CU-31, Baja de sanitario) , controlarán los sanitarios del hospital, así como sus rangos y obligaciones..
		\tabularnewline
		\hline
	\end{tabular}

	\vspace{0.5cm}
	%Resumen
	\begin{tabular}{|>{\raggedright}p{337pt}|}
		\hline
		\textbf{Resumen}\tabularnewline
		\hline
Con esta funcionalidad, se controla el personal que atiende a nuestros pacientes y nos aseguramos de su capacitación.
		\tabularnewline
		\hline
	\end{tabular}
	\vspace{0.5cm}

	%Plantilla Extendida

	\begin{tabular}{|>{\raggedright}p{11pt}|>{\raggedright}p{138pt}|>{\raggedright}p{10pt}|>{\raggedright}p{140pt}|}
		\hline
		\multicolumn{4}{|p{301pt}|}{
		\textbf{Curso normal (básico)}}\tabularnewline
		\hline
		\centering 1 & Compañía Médica: Evalúa la cualificación del sanitario. & 2 \centering  & Notifica al sanitario del resultado de su evaluación. \tabularnewline
		\hline
		\centering 3 & Sanitario: Recibe la notificación de su valoración como positiva. & \centering  &  \tabularnewline
		\hline
	\end{tabular}

	\vspace{0.5cm}
	\newpage

	%CU Alternos
	\begin{tabular}{|>{\raggedright}p{11pt}|>{\raggedright}p{56pt}|>{\raggedright}p{91pt}|>{\raggedright}p{46pt}|>{\raggedright}p{83pt}|}
		\hline
		\multicolumn{5}{|p{337pt}|}{\textbf{Cursos alternos}}\tabularnewline
		\hline
		\centering \textbf{3a} & \multicolumn{4}{p{300pt}|}{ \textbf{3 $\rightarrow$ 3a}
    Se procede al curso de contratación.}\tabularnewline
		\hline
	\end{tabular}
	\vspace{0.5cm}

	%Otros Datos
	\begin{tabular}{|>{\raggedright}p{11pt}|>{\raggedright}p{56pt}|>{\raggedright}p{88pt}|>{\raggedright}p{50pt}|>{\raggedright}p{83pt}|}
		\hline
		\multicolumn{5}{|p{337pt}|}{\textbf{Otros datos}}\tabularnewline
		\hline

		 \multicolumn{2}{|p{68pt}|}{
		\textbf{Frecuencia \newline esperada}} & Tantas veces como se hayan jubilado, despedido o trasladado personal al año. \quad & \textbf{Rendimiento} &
		Medio \tabularnewline
		\hline


		 \multicolumn{2}{|p{68pt}|}{
		\textbf{Importancia}} & Alta \quad  & \textbf{Urgencia} &
Alta \tabularnewline
		\hline
		\multicolumn{2}{|p{68pt}|}{\textbf{Estado}} & En espera \quad  & \textbf{Estabilidad} &
		Media \tabularnewline
		\hline
	\end{tabular}

	%Comentarios
	\vspace{0.5cm}
	\begin{tabular}{|>{\raggedright}p{337pt}|}
		\hline
		\multicolumn{1}{|p{337pt}|}{\textbf{Comentarios}}\tabularnewline
		\hline
		\multicolumn{1}{|p{337pt}|}{} \tabularnewline
		\hline
	\end{tabular}
