

\begin{tabular}{|>{\raggedright}p{58pt}|>{\raggedright}p{109pt}|>{\raggedright}p{1pt}|>{\raggedright}p{17pt}|>{\raggedright}p{28pt}|>{\raggedright}p{0pt}|>{\raggedright}p{18pt}|>{\raggedright}p{20pt}|}
	\hline
	 \textbf{Caso de Uso} &

	%Nombre del CU e identificador
	\multicolumn{5}{p{155pt}|}{Informar prueba.}	& \multicolumn{2}{p{39pt}|}{\textbf{CU-53}}\tabularnewline

	\hline

	%Listado de actores
	\textbf{Actores} & \multicolumn{7}{p{194pt}|}{Médico (I)}\tabularnewline
	\hline

	%Tipo de CU
	\textbf{Tipo} & \multicolumn{7}{p{194pt}|}{Primario y esencial.}\tabularnewline
	\hline

	%Requisitos del CU
	\textbf{Referencias} & \multicolumn{2}{p{110pt}|}{} & \multicolumn{5}{p{84pt}|}{}\tabularnewline
	\hline

	%Precondiciones
	\textbf{Precondición} & \multicolumn{7}{p{194pt}|}{Se debe haber realizado una prueba a un paciente sobre la que realizar el informe.}\tabularnewline
	\hline

	%Postcondiciones
	\textbf{Postcondición} & \multicolumn{7}{p{194pt}|}{Se dispone en el sistema del informe resultante del diagnóstico realizado.}\tabularnewline
	\hline

	%Autor y fecha
	\textbf{Autor} & Luis Antonio Ortega Andrés & \multicolumn{2}{p{30pt}|}{
	\textbf{Fecha}} & 10 de abril & \multicolumn{2}{p{30pt}|}{
	\textbf{Versión}} & 1.0 \tabularnewline
	\hline
	\end{tabular}

	\vspace{0.5cm}

	%Proposito
	\begin{tabular}{|>{\raggedright}p{337pt}|}
		\hline
		\textbf{Propósito} \tabularnewline \hline
			Disponer en el sistema del informe del diagnostico realizado a cada paciente.
		\tabularnewline
		\hline
	\end{tabular}

	\vspace{0.5cm}
	%Resumen
	\begin{tabular}{|>{\raggedright}p{337pt}|}
		\hline
		\textbf{Resumen}\tabularnewline
		\hline
			Se debe realizar un informe de cada prueba por parte del médico responsable, esto será necesario para la posterior recogida de pruebas por parte del paciente, además del registor resultante en el historial clínico.
		\tabularnewline
		\hline
	\end{tabular}
	\vspace{0.5cm}

	%Plantilla Extendida

	\begin{tabular}{|>{\raggedright}p{11pt}|>{\raggedright}p{138pt}|>{\raggedright}p{10pt}|>{\raggedright}p{140pt}|}
		\hline
		\multicolumn{4}{|p{301pt}|}{
		\textbf{Curso normal (básico)}}\tabularnewline
		\hline
		\centering 1 & Médico: Realiza el informe de las pruebas. & \centering  & \tabularnewline
		\hline
		\centering 2 & Médico: Envía informe al sistema. & \centering 3 &  Recibe y almacena el informe.\tabularnewline
		\hline
	\end{tabular}

	\vspace{0.5cm}
	\newpage

	%CU Alternos
	\begin{tabular}{|>{\raggedright}p{11pt}|>{\raggedright}p{56pt}|>{\raggedright}p{91pt}|>{\raggedright}p{46pt}|>{\raggedright}p{83pt}|}
		\hline
		\multicolumn{5}{|p{337pt}|}{\textbf{Cursos alternos}}\tabularnewline
		\hline
		\centering \textbf{2a} & \multicolumn{4}{p{278pt}|}{
		 Debido a errores del sistema, no es posible la subida del informe.}\tabularnewline
		\hline
		\centering \textbf{2b} & \multicolumn{4}{p{278pt}|}{
		El informe presenta errores fatales y no es posible su correcto archivado.
		}\tabularnewline
		\hline
	\end{tabular}
	\vspace{0.5cm}

	%Otros Datos
	\begin{tabular}{|>{\raggedright}p{11pt}|>{\raggedright}p{56pt}|>{\raggedright}p{88pt}|>{\raggedright}p{50pt}|>{\raggedright}p{83pt}|}
		\hline
		\multicolumn{5}{|p{337pt}|}{\textbf{Otros datos}}\tabularnewline
		\hline
		
		 \multicolumn{2}{|p{68pt}|}{
		\textbf{Frecuencia esperada}} & Número actual de pacientes al año.\quad & \textbf{Rendimiento} & 
		No más de un día.\tabularnewline
		\hline
		
		
		 \multicolumn{2}{|p{68pt}|}{
		\textbf{Importancia}} & Alta \quad  & \textbf{Urgencia} & Moderada \tabularnewline
		\hline
		\multicolumn{2}{|p{68pt}|}{\textbf{Estado}} & - & \textbf{Estabilidad} &
		Alta \tabularnewline
		\hline
	\end{tabular}

	%Comentarios
	\vspace{0.5cm}
	\begin{tabular}{|>{\raggedright}p{337pt}|}
		\hline
		\multicolumn{1}{|p{337pt}|}{\textbf{Comentarios}}\tabularnewline
		\hline
		\multicolumn{1}{|p{337pt}|}{La frecuencia media del caso de uso es de una petición de resultados por cada vez que un paciente se realiza una prueba, tomando como estimación que esto ocurre de forma anual.} \tabularnewline
		\hline
	\end{tabular}
