

\begin{tabular}{|>{\raggedright}p{58pt}|>{\raggedright}p{109pt}|>{\raggedright}p{1pt}|>{\raggedright}p{17pt}|>{\raggedright}p{28pt}|>{\raggedright}p{0pt}|>{\raggedright}p{18pt}|>{\raggedright}p{20pt}|}
	\hline
	 \textbf{Caso de Uso} &

	%Nombre del CU e identificador
	\multicolumn{5}{p{155pt}|}{Pedir cita remoto}	& \multicolumn{2}{p{39pt}|}{\textbf{CU-44}}\tabularnewline

	\hline

	%Listado de actores
	\textbf{Actores} & \multicolumn{7}{p{194pt}|}{Paciente (I)}\tabularnewline
	\hline

	%Tipo de CU
	\textbf{Tipo} & \multicolumn{7}{p{194pt}|}{Primario y esencial}\tabularnewline
	\hline

	%Requisitos del CU
	\textbf{Referencias} & \multicolumn{2}{p{110pt}|}{-} & \multicolumn{5}{p{84pt}|}{-}\tabularnewline
	\hline

	%Precondiciones
	\textbf{Precondición} & \multicolumn{7}{p{194pt}|}{Se debe realizar de forma correcta la petición de cita médica por parte del paciente.}\tabularnewline
	\hline

	%Postcondiciones
	\textbf{Postcondición} & \multicolumn{7}{p{194pt}|}{Se debe registrar de forma inmediata la cita médica pedida por el paciente.}\tabularnewline
	\hline

	%Autor y fecha
	\textbf{Autor} & Luis Antonio Ortega Andrés & \multicolumn{2}{p{30pt}|}{
	\textbf{Fecha}} & 10 de abril & \multicolumn{2}{p{30pt}|}{
	\textbf{Versión}} & 1.0 \tabularnewline
	\hline
	\end{tabular}

	\vspace{0.5cm}

	%Proposito
	\begin{tabular}{|>{\raggedright}p{337pt}|}
		\hline
		\textbf{Propósito} \tabularnewline \hline
			Permitir al paciente realizar una petición de cita médita de forma remota.
		\tabularnewline
		\hline
	\end{tabular}

	\vspace{0.5cm}
	%Resumen
	\begin{tabular}{|>{\raggedright}p{337pt}|}
		\hline
		\textbf{Resumen}\tabularnewline
		\hline
			Los pacientes deben tener la posibilidad de realizar una petición de cita médica de forma remota, y ante esta petición el sistema debe ser capaz de registrarlo de forma inmediata y eficiente.
		\tabularnewline
		\hline
	\end{tabular}
	\vspace{0.5cm}

	%Plantilla Extendida

	\begin{tabular}{|>{\raggedright}p{11pt}|>{\raggedright}p{138pt}|>{\raggedright}p{10pt}|>{\raggedright}p{140pt}|}
		\hline
		\multicolumn{4}{|p{301pt}|}{
		\textbf{Curso normal (básico)}}\tabularnewline
		\hline
		\centering 1 & Paciente: Realiza la petición de cita remota. & \centering  &  Registra la cita del paciente. \tabularnewline
		\hline
	\end{tabular}

	\vspace{0.5cm}
	
	%CU Alternos
	\begin{tabular}{|>{\raggedright}p{11pt}|>{\raggedright}p{56pt}|>{\raggedright}p{91pt}|>{\raggedright}p{46pt}|>{\raggedright}p{83pt}|}
		\hline
		\multicolumn{5}{|p{337pt}|}{\textbf{Cursos alternos}}\tabularnewline
		\hline
		\centering \textbf{1a} & \multicolumn{4}{p{278pt}|}{
		 El paciente intenta registrar una cita ya concertada, luego no es posible realizar la acción.}\tabularnewline
		\hline
	\end{tabular}
	\vspace{0.5cm}

	%Otros Datos
	\begin{tabular}{|>{\raggedright}p{11pt}|>{\raggedright}p{56pt}|>{\raggedright}p{88pt}|>{\raggedright}p{50pt}|>{\raggedright}p{83pt}|}
		\hline
		\multicolumn{5}{|p{337pt}|}{\textbf{Otros datos}}\tabularnewline
		\hline
		
		 \multicolumn{2}{|p{68pt}|}{
		\textbf{Frecuencia esperada}} & Número de pacientes actuales por año. \quad & \textbf{Rendimiento} & 
		Casi inmediato. \tabularnewline
		\hline
		
		
		 \multicolumn{2}{|p{68pt}|}{
		\textbf{Importancia}} & Alta. \quad  & \textbf{Urgencia} &
		Alta. \tabularnewline
		\hline
		\multicolumn{2}{|p{68pt}|}{\textbf{Estado}} & - \quad  & \textbf{Estabilidad} &
		Alta \tabularnewline
		\hline
	\end{tabular}

	%Comentarios
	\vspace{0.5cm}
	\begin{tabular}{|>{\raggedright}p{337pt}|}
		\hline
		\multicolumn{1}{|p{337pt}|}{\textbf{Comentarios}}\tabularnewline
		\hline
		\multicolumn{1}{|p{337pt}|}{Sobre la frecuencia esperada, la frecuencia esperada de peciciones de citas en general es que cada paciente concierte una cita al año, luego la cantidad de citas remotas debe ser menor.} \tabularnewline
		\hline
	\end{tabular}
