

\begin{tabular}{|>{\raggedright}p{58pt}|>{\raggedright}p{109pt}|>{\raggedright}p{1pt}|>{\raggedright}p{17pt}|>{\raggedright}p{28pt}|>{\raggedright}p{0pt}|>{\raggedright}p{18pt}|>{\raggedright}p{20pt}|}
	\hline
	 \textbf{Caso de Uso} &

	%Nombre del CU e identificador
	\multicolumn{5}{p{155pt}|}{Asignar horario personal}	& \multicolumn{2}{p{39pt}|}{\textbf{CU-26}}\tabularnewline

	\hline

	%Listado de actores
	\textbf{Actores} & \multicolumn{7}{p{194pt}|}{Administrativo}\tabularnewline
	\hline

	%Tipo de CU
	\textbf{Tipo} & \multicolumn{7}{p{194pt}|}{Primario y esencial}\tabularnewline
	\hline

	%Requisitos del CU
	\textbf{Referencias} & \multicolumn{2}{p{110pt}|}{} & \multicolumn{5}{p{84pt}|}{}\tabularnewline
	\hline

	%Precondiciones
	\textbf{Precondición} & \multicolumn{7}{p{194pt}|}{Debe existir el trabajador al que se asigna el horario}\tabularnewline
	\hline

	%Postcondiciones
	\textbf{Postcondición} & \multicolumn{7}{p{194pt}|}{El horario del trabajador pasa a ser el asignado.}\tabularnewline
	\hline

	%Autor y fecha
	\textbf{Autor} & Daniel Pozo Escalona & \multicolumn{2}{p{30pt}|}{
	\textbf{Fecha}} & 10 de abril & \multicolumn{2}{p{30pt}|}{
	\textbf{Versión}} & 1.0 \tabularnewline
	\hline
	\end{tabular}

	\vspace{0.5cm}

	%Proposito
	\begin{tabular}{|>{\raggedright}p{337pt}|}
		\hline
		\textbf{Propósito} \tabularnewline \hline
		Gestionar el personal.
		\tabularnewline
		\hline
	\end{tabular}

	\vspace{0.5cm}
	%Resumen
	\begin{tabular}{|>{\raggedright}p{337pt}|}
		\hline
		\textbf{Resumen}\tabularnewline
		\hline
		Cuando se requiera, se asigna un horario a un trabajador.
		\tabularnewline
		\hline
	\end{tabular}
	\vspace{0.5cm}

	%Plantilla Extendida

	\begin{tabular}{|>{\raggedright}p{11pt}|>{\raggedright}p{138pt}|>{\raggedright}p{10pt}|>{\raggedright}p{140pt}|}
		\hline
		\multicolumn{4}{|p{301pt}|}{
		\textbf{Curso normal (básico)}}\tabularnewline
		\hline
		\centering 1 & Administrativo: solicita modificar el horario de un trabajador. & \centering  & \tabularnewline
		\hline
		\centering 2 & Administrativo: introduce el horario del trabajador. & \centering 3 & Verifica la corrección de los datos y los introduce en el sistema. \tabularnewline
		\hline
		%% \centering  &  & \centering 4  & Crea una nueva cuenta con los datos del paciente. \tabularnewline
		%% \hline
		%% \centering 5 & Paciente: Recibe los resultados. & \centering  & \tabularnewline
		%% \hline
	\end{tabular}

	\vspace{0.5cm}

	%CU Alternos
	%% \begin{tabular}{|>{\raggedright}p{11pt}|>{\raggedright}p{56pt}|>{\raggedright}p{91pt}|>{\raggedright}p{46pt}|>{\raggedright}p{83pt}|}
	%% 	\hline
	%% 	\multicolumn{5}{|p{337pt}|}{\textbf{Cursos alternos}}\tabularnewline
	%% 	\hline
	%% 	\centering \textbf{3a} & \multicolumn{4}{p{278pt}|}{
	%% No se crea la cuenta por estar registrado el paciente.}\tabularnewline
	%% 	\hline
	%% 	%% \centering \textbf{3b} & \multicolumn{4}{p{278pt}|}{
	%% 	%% No se concede el acceso a los resultados debido a que no existe la prueba que el paciente indica que se le ha realizado.
	%% 	%% }\tabularnewline
	%% 	%% \hline
	%% \end{tabular}
	%% \vspace{0.5cm}

	%Otros Datos
	\begin{tabular}{|>{\raggedright}p{11pt}|>{\raggedright}p{56pt}|>{\raggedright}p{88pt}|>{\raggedright}p{50pt}|>{\raggedright}p{83pt}|}
		\hline
		\multicolumn{5}{|p{337pt}|}{\textbf{Otros datos}}\tabularnewline
		\hline

		 \multicolumn{2}{|p{68pt}|}{
		\textbf{Frecuencia esperada}} & Tantas veces como trabajadores por cuatro al año.\quad & \textbf{Rendimiento} &
		Instantáneo.\tabularnewline
		\hline


		 \multicolumn{2}{|p{68pt}|}{
		\textbf{Importancia}} & Media \quad  & \textbf{Urgencia} & Moderada \tabularnewline
		\hline
		\multicolumn{2}{|p{68pt}|}{\textbf{Estado}} & - & \textbf{Estabilidad} &
		Alta \tabularnewline
		\hline
	\end{tabular}

	%Comentarios
	\vspace{0.5cm}
	\begin{tabular}{|>{\raggedright}p{337pt}|}
		\hline
		\multicolumn{1}{|p{337pt}|}{\textbf{Comentarios}}\tabularnewline
		\hline
		\multicolumn{1}{|p{337pt}|}{} \tabularnewline
		\hline
	\end{tabular}
