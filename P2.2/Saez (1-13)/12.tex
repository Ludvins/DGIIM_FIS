\documentclass[10pt,a4paper,spanish]{report}

\usepackage[spanish]{babel}
\usepackage[utf8]{inputenc}
\usepackage{amsmath, amsthm}
\usepackage{amsfonts, amssymb, latexsym}
\usepackage{enumerate}
\usepackage[official]{eurosym}
\usepackage{graphicx}
\usepackage[usenames, dvipsnames]{color}
\usepackage{colortbl}
\usepackage{multirow}
\usepackage{fancyhdr}
\usepackage{fancybox}
\usepackage{pseudocode}
\usepackage[all]{xy}
\usepackage{hyperref}
\usepackage{tikz}
\usepackage{pgfplots}

\pgfplotsset{compat=1.5}

% a4large.sty -- fill an A4 (210mm x 297mm) page
% Note: 1 inch = 25.4 mm = 72.27 pt
%       1 pt = 3.5 mm (approx)

% vertical page layout -- one inch margin top and bottom
\topmargin      0 mm    % top margin less 1 inch
\headheight     0 mm    % height of box containing the head
\headsep       10 mm    % space between the head and the body of the page
\textheight   250 mm
\footskip      14 mm    % distance from bottom of body to bottom of foot

% horizontal page layout -- one inch margin each side
%\oddsidemargin    0   mm    % inner margin less one inch on odd pages
%\evensidemargin   0   mm    % inner margin less one inch on even pages
%\textwidth      159.2 mm    % normal width of text on page

\usepackage[math]{iwona}
\usepackage[T1]{fontenc}
\usepackage{inconsolata}

\definecolor{webgreen}{rgb}{0, 0.5, 0} % less intense green 
\definecolor{webblue}{rgb}{0, 0, 0.5}  % less intense blue
\definecolor{webred}{rgb}{0.5, 0, 0}   % less intense red
\definecolor{dblackcolor}{rgb}{0.0,0.0,0.0}
\definecolor{dbluecolor}{rgb}{.01,.02,0.7}
\definecolor{dredcolor}{rgb}{0.8,0,0}
\definecolor{dgraycolor}{rgb}{0.30,0.3,0.30}

\newcommand{\HRule}{\rule{\linewidth}{0.5mm}}

\pagestyle{fancy}

\renewcommand{\chaptermark}[1]{%
\markboth{#1}{}}
\renewcommand{\sectionmark}[1]{%
\markright{\thesection\ #1}}
\fancyhf{}
\fancyhead[LE,RO]{\bfseries\thepage}
\fancyhead[LO]{\bfseries\leftmark}
\renewcommand{\headrulewidth}{0.5pt}
\renewcommand{\footrulewidth}{0pt}
\addtolength{\headheight}{0.5pt}
\fancypagestyle{plain}{
\fancyhead{}
\renewcommand{\headrulewidth}{0pt}
}

\usepackage{sectsty}
\chapterfont{\fontfamily{pag}\selectfont} %% for chapter if you want
\sectionfont{\fontfamily{pag}\selectfont}
\subsectionfont{\fontfamily{pag}\selectfont}
\subsubsectionfont{\fontfamily{pag}\selectfont}

\renewcommand{\labelenumi}{\arabic{enumi}. }
\renewcommand{\labelenumii}{\labelenumi\alph{enumii}) }
\renewcommand{\labelenumiii}{\labelenumii\roman{enumiii}: }

\begin{document}

\begin{tabular}{|>{\raggedright}p{58pt}|>{\raggedright}p{109pt}|>{\raggedright}p{1pt}|>{\raggedright}p{17pt}|>{\raggedright}p{28pt}|>{\raggedright}p{0pt}|>{\raggedright}p{18pt}|>{\raggedright}p{20pt}|}
	\hline
	 \textbf{Caso de Uso} &

	%Nombre del CU e identificador
	\multicolumn{5}{p{155pt}|}{Llamar al siguiente paciente}	& \multicolumn{2}{p{39pt}|}{\textbf{CU12}}\tabularnewline

	\hline

	%Listado de actores
	\textbf{Actores} & \multicolumn{7}{p{194pt}|}{Paciente, Sanitario}\tabularnewline
	\hline

	%Tipo de CU
	\textbf{Tipo} & \multicolumn{7}{p{194pt}|}{Primario, Real}\tabularnewline
	\hline

	%Requisitos del CU
	\textbf{Referencias} & \multicolumn{2}{p{110pt}|}{La cita anterior deberá haber terminado} & \multicolumn{5}{p{84pt}|}{\textbf{CU11}:Terminar consulta, \textbf{CU1:Iniciar consulta}}\tabularnewline
	\hline

	%Precondiciones
	\textbf{Precondición} & \multicolumn{7}{p{194pt}|}{Condiciones sobre el estado del sistema que tienen	que ser ciertas para que se pueda realizar el CU}\tabularnewline
	\hline

	%Postcondiciones
	\textbf{Postcondición} & \multicolumn{7}{p{194pt}|}{La nueva cita estará lista para comenzar}\tabularnewline
	\hline

	%Autor y fecha
	\textbf{Autor} & Javier Sáez  & \multicolumn{2}{p{30pt}|}{
	\textbf{Fecha}} & \today & \multicolumn{2}{p{30pt}|}{
	\textbf{Versión}} & 1.0 \tabularnewline
	\hline
	\end{tabular}

	\vspace{0.5cm}

	%Proposito
	\begin{tabular}{|>{\raggedright}p{337pt}|}
		\hline
		\textbf{Propósito} \tabularnewline \hline
			Que comience la cita con el siguiente paciente
		\tabularnewline
		\hline
	\end{tabular}

	\vspace{0.5cm}
	%Resumen
	\begin{tabular}{|>{\raggedright}p{337pt}|}
		\hline
		\textbf{Resumen}\tabularnewline
		\hline
			Tras haber abandonado la sala el último paciente, el sistema indicará en una pantalla y mediante sonido quién es el siguiente paciente que debe entrar en la sala para ser atendido. Este se dirigirá a la sala donde será atendido y comienza su cita.
		\tabularnewline
		\hline
	\end{tabular}
	\vspace{0.5cm}

	%Plantilla Extendida

	\begin{tabular}{|>{\raggedright}p{11pt}|>{\raggedright}p{138pt}|>{\raggedright}p{10pt}|>{\raggedright}p{140pt}|}
		\hline
		\multicolumn{4}{|p{301pt}|}{
		\textbf{Curso normal (básico)}}\tabularnewline
		\hline
		\centering &  & \centering  1& Busca quién es el siguiente paciente en su base de datos de citas\tabularnewline
		\hline
		\centering  &   & \centering 2 &  Comunica mediante audio y en una pantalla quién es el siguiente en tener una cita\tabularnewline
		\hline
		\centering  3&  Paciente: Ve que es su turno y se dirige hacia la sala indicada por el sistema& \centering  & \tabularnewline
		\hline
	\end{tabular}

	\vspace{0.5cm}
	\newpage

	%CU Alternos
	\begin{tabular}{|>{\raggedright}p{11pt}|>{\raggedright}p{56pt}|>{\raggedright}p{91pt}|>{\raggedright}p{46pt}|>{\raggedright}p{83pt}|}
		\hline
		\multicolumn{5}{|p{337pt}|}{\textbf{Cursos alternos}}\tabularnewline
		\hline
		\centering \textbf{3a} & \multicolumn{4}{p{278pt}|}{
		El paciente no está o no se da cuenta de que es su turno}\tabularnewline
		\hline
		\centering \textbf{3b} & \multicolumn{4}{p{278pt}|}{ El sistema quita de la cola de pacientes al paciente actual y busca al siguiente
		}\tabularnewline
		\hline
		\centering  \textbf{3c}& \multicolumn{4}{p{278pt}|}{Vuelve al paso \textbf{2} del curso normal del CU.}\tabularnewline
		\hline
	\end{tabular}
	\vspace{0.5cm}

	%Otros Datos
	\begin{tabular}{|>{\raggedright}p{11pt}|>{\raggedright}p{56pt}|>{\raggedright}p{88pt}|>{\raggedright}p{50pt}|>{\raggedright}p{83pt}|}
		\hline
		\multicolumn{5}{|p{337pt}|}{\textbf{Otros datos}}\tabularnewline
		\hline
		
		 \multicolumn{2}{|p{68pt}|}{
		\textbf{Frecuencia esperada}} & Moderada \quad & \textbf{Rendimiento} & 
		Alto \tabularnewline
		\hline
		
		
		 \multicolumn{2}{|p{68pt}|}{
		\textbf{Importancia}} & Alta\quad  & \textbf{Urgencia} &
		Moderada \tabularnewline
		\hline
		\multicolumn{2}{|p{68pt}|}{\textbf{Estado}} & Activo\quad  & \textbf{Estabilidad} &
		Alta\tabularnewline
		\hline
	\end{tabular}

	

\end{document}

