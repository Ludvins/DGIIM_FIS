\begin{tabular}{|>{\raggedright}p{58pt}|>{\raggedright}p{109pt}|>{\raggedright}p{1pt}|>{\raggedright}p{17pt}|>{\raggedright}p{28pt}|>{\raggedright}p{0pt}|>{\raggedright}p{18pt}|>{\raggedright}p{20pt}|}
	\hline
	 \textbf{Caso de Uso} &

	%Nombre del CU e identificador
	\multicolumn{5}{p{155pt}|}{Nuevo recurso
}	& \multicolumn{2}{p{39pt}|}{\textbf{Identificador} CU-32}\tabularnewline

	\hline

	%Listado de actores
	\textbf{Actores} & \multicolumn{7}{p{194pt}|}{Administrativo(I)}\tabularnewline
	\hline

	%Tipo de CU
	\textbf{Tipo} & \multicolumn{7}{p{194pt}|}{Primario \textbar Esencial}\tabularnewline
	\hline

	%Requisitos del CU
	\textbf{Referencias} & \multicolumn{2}{p{110pt}|}{El registro debe de ser correcto.} & \multicolumn{5}{p{84pt}|}{Modificar Recurso, Baja de recurso, Asignación permanente de recurso, Eliminar asignación de recurso, Asignación puntual de recurso.}\tabularnewline
	\hline

	%Precondiciones
	\textbf{Precondición} & \multicolumn{7}{p{194pt}|}{El recurso debe de existir.}\tabularnewline
	\hline

	%Postcondiciones
	\textbf{Postcondición} & \multicolumn{7}{p{194pt}|}{El recurso quedará registrado en el sistema. }\tabularnewline
	\hline

	%Autor y fecha
	\textbf{Autor} & Laura Gómez Garrido  & \multicolumn{2}{p{30pt}|}{
	\textbf{Fecha}} & \today & \multicolumn{2}{p{30pt}|}{
	\textbf{Versión}} & 1.0 \tabularnewline
	\hline
	\end{tabular}

	\vspace{0.5cm}

	%Proposito
	\begin{tabular}{|>{\raggedright}p{337pt}|}
		\hline
		\textbf{Propósito} \tabularnewline \hline
			Con esta funcionalidad, creamos un nuevo recurso al cual tendremos acceso a través de nuestro sistema.
		\tabularnewline
		\hline
	\end{tabular}

	\vspace{0.5cm}
	%Resumen
	\begin{tabular}{|>{\raggedright}p{337pt}|}
		\hline
		\textbf{Resumen}\tabularnewline
		\hline
    Junto a (CU-33, Modificar recurso),(CU-34, Baja de recurso), (CU-35, Asignación permanente de recurso), (CU-36, Eliminar asignación de recurso),(CU-37, Asignación puntual de recurso) , se controlan todos los recursos del sistema.
		\tabularnewline
		\hline
	\end{tabular}
	\vspace{0.5cm}

	%Plantilla Extendida

	\begin{tabular}{|>{\raggedright}p{11pt}|>{\raggedright}p{138pt}|>{\raggedright}p{10pt}|>{\raggedright}p{140pt}|}
		\hline
		\multicolumn{4}{|p{301pt}|}{
		\textbf{Curso normal (básico)}}\tabularnewline
		\hline
		\centering 1 & Administrativo: Añade el nuevo recurso al sistema & 2 \centering  & Guarda el nuevo recurso entre los existentes. \tabularnewline
    \hline
    \centering &  & 3 \centering  & Notifica de la existencia del nuevo recurso. \tabularnewline
    \hline
	\end{tabular}

	\vspace{0.5cm}


	%CU Alternos
	\begin{tabular}{|>{\raggedright}p{11pt}|>{\raggedright}p{56pt}|>{\raggedright}p{91pt}|>{\raggedright}p{46pt}|>{\raggedright}p{83pt}|}
		\hline
		\multicolumn{5}{|p{337pt}|}{\textbf{Cursos alternos}}\tabularnewline
		\hline
		\centering \textbf{3a} & \multicolumn{4}{p{300pt}|}{ \textbf{3 $\rightarrow$ 3a}
    Da la posibilidad de utilizar todos los demás casos de uso.}\tabularnewline
		\hline
	\end{tabular}
	\vspace{0.5cm}

	%Otros Datos
	\begin{tabular}{|>{\raggedright}p{11pt}|>{\raggedright}p{56pt}|>{\raggedright}p{88pt}|>{\raggedright}p{50pt}|>{\raggedright}p{83pt}|}
		\hline
		\multicolumn{5}{|p{337pt}|}{\textbf{Otros datos}}\tabularnewline
		\hline

		 \multicolumn{2}{|p{68pt}|}{
		\textbf{Frecuencia \newline esperada}} & Entre 10 y 20 veces al año. \quad & \textbf{Rendimiento} &
		Alto \tabularnewline
		\hline


		 \multicolumn{2}{|p{68pt}|}{
		\textbf{Importancia}} & Alta \quad  & \textbf{Urgencia} &
Media \tabularnewline
		\hline
		\multicolumn{2}{|p{68pt}|}{\textbf{Estado}} & En espera \quad  & \textbf{Estabilidad} &
		Baja \tabularnewline
		\hline
	\end{tabular}

	%Comentarios
	\vspace{0.5cm}
	\begin{tabular}{|>{\raggedright}p{337pt}|}
		\hline
		\multicolumn{1}{|p{337pt}|}{\textbf{Comentarios}}\tabularnewline
		\hline
		\multicolumn{1}{|p{337pt}|}{Por recurso se entienden tanto las salas y espacios(despachos de consulta, salas de aparatos, etc.), como la maquinaria (aparatos de rayos X, etc.)} \tabularnewline
		\hline
	\end{tabular}
