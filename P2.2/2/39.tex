
\begin{tabular}{|>{\raggedright}p{58pt}|>{\raggedright}p{109pt}|>{\raggedright}p{1pt}|>{\raggedright}p{17pt}|>{\raggedright}p{28pt}|>{\raggedright}p{0pt}|>{\raggedright}p{18pt}|>{\raggedright}p{20pt}|}
	\hline
	 \textbf{Caso de Uso} &

	%Nombre del CU e identificador
	\multicolumn{5}{p{155pt}|}{Pedir cita
}	& \multicolumn{2}{p{39pt}|}{\textbf{Identificador} CU-39}\tabularnewline

	\hline

	%Listado de actores
	\textbf{Actores} & \multicolumn{7}{p{194pt}|}{Administrativo, Paciente(I)}\tabularnewline
	\hline

	%Tipo de CU
	\textbf{Tipo} & \multicolumn{7}{p{194pt}|}{Primario \textbar Esencial}\tabularnewline
	\hline

	%Requisitos del CU
	\textbf{Referencias} & \multicolumn{2}{p{110pt}|}{La cita debe de ser reflejada correctamente.} & \multicolumn{5}{p{84pt}|}{Asistir a cita, Pedir cambio de cita, Anular Cita}\tabularnewline
	\hline

	%Precondiciones
	\textbf{Precondición} & \multicolumn{7}{p{194pt}|}{}\tabularnewline
	\hline

	%Postcondiciones
	\textbf{Postcondición} & \multicolumn{7}{p{194pt}|}{La cita debe de haber sido asignada correctamente.}\tabularnewline
	\hline

	%Autor y fecha
	\textbf{Autor} & Laura Gómez Garrido  & \multicolumn{2}{p{30pt}|}{
	\textbf{Fecha}} & \today & \multicolumn{2}{p{30pt}|}{
	\textbf{Versión}} & 1.0 \tabularnewline
	\hline
	\end{tabular}

	\vspace{0.5cm}

	%Proposito
	\begin{tabular}{|>{\raggedright}p{337pt}|}
		\hline
		\textbf{Propósito} \tabularnewline \hline
		Asignar una cita con el sanitario adecuado a un paciente que lo requiere.
		\tabularnewline
		\hline
	\end{tabular}

	\vspace{0.5cm}
	%Resumen
	\begin{tabular}{|>{\raggedright}p{337pt}|}
		\hline
		\textbf{Resumen}\tabularnewline
		\hline
    Cuando un paciente padece de una enfermedad o creer padecerla, lo normal es que decida asistir al hospital para recibir el diagnóstico y tratamiento adecuado. Para una correcta atención, deberá de pedir cita.
	\tabularnewline
		\hline
	\end{tabular}
	\vspace{0.5cm}

	%Plantilla Extendida

	\begin{tabular}{|>{\raggedright}p{11pt}|>{\raggedright}p{138pt}|>{\raggedright}p{10pt}|>{\raggedright}p{140pt}|}
		\hline
		\multicolumn{4}{|p{301pt}|}{
		\textbf{Curso normal (básico)}}\tabularnewline
		\hline
		\centering 1 & Paciente: Solicita una cita & 2 \centering  & Proporciona un conjunto de citas disponibles así cómo el sanitario asignado a dichas citas, para que el usuario elija entre la que más le convenga. \tabularnewline
    \hline
    \centering 3 & Paciente: Selecciona la cita que prefiere.  & 4 \centering  & Almacena la cita y actualiza el listado de citas disponibles.  \tabularnewline
    \hline
	\end{tabular}

	\vspace{0.5cm}


	%CU Alternos
	\begin{tabular}{|>{\raggedright}p{11pt}|>{\raggedright}p{56pt}|>{\raggedright}p{91pt}|>{\raggedright}p{46pt}|>{\raggedright}p{83pt}|}
		\hline
		\multicolumn{5}{|p{337pt}|}{\textbf{Cursos alternos}}\tabularnewline
		\hline
		\centering \textbf{1a} & \multicolumn{4}{p{300pt}|}{ \textbf{1a $\rightarrow$ 1b}
     El paciente contacta con el administrativo para concertar la cita.}\tabularnewline
		\hline
    \centering \textbf{1b} & \multicolumn{4}{p{300pt}|}{ \textbf{1a $\rightarrow$ 1b $\rightarrow$ 4}
     El administrativo concerta la cita de acuerdo a las preferencias del paciente y a las citas disponibles.}\tabularnewline
    \hline
	\end{tabular}
	\vspace{0.5cm}



	%Otros Datos
	\begin{tabular}{|>{\raggedright}p{11pt}|>{\raggedright}p{56pt}|>{\raggedright}p{88pt}|>{\raggedright}p{50pt}|>{\raggedright}p{83pt}|}
		\hline
		\multicolumn{5}{|p{337pt}|}{\textbf{Otros datos}}\tabularnewline
		\hline

		 \multicolumn{2}{|p{68pt}|}{
		\textbf{Frecuencia \newline esperada}} & Decenas de veces al día. \quad & \textbf{Rendimiento} &
		Alto \tabularnewline
		\hline


		 \multicolumn{2}{|p{68pt}|}{
		\textbf{Importancia}} & Alta \quad  & \textbf{Urgencia} &
Media \tabularnewline
		\hline
		\multicolumn{2}{|p{68pt}|}{\textbf{Estado}} & Activo \quad  & \textbf{Estabilidad} &
		Alta \tabularnewline
		\hline
	\end{tabular}

	%Comentarios
	%\vspace{0.5cm}
	%\begin{tabular}{|>{\raggedright}p{337pt}|}
		%\hline
		%\multicolumn{1}{|p{337pt}|}{\textbf{Comentarios}}\tabularnewline
		%\hline
		%\multicolumn{1}{|p{337pt}|}{} \tabularnewline
		%\hline
	%\end{tabular}
