 \begin{table}[H]
  \centering
  \begin{tabularx}{\textwidth}{l|X}
    \textbf{Nombre}        &  infoCita = pedirCambioCita ( idCita, fecha, hora ) \\ 
    \textbf{Responsabilidad}  &  Crea una nueva cita a la fecha y hora
    especificadas, sustituyendo la cita existente identificada por idCita. \\ 
    \textbf{Tipo}        &  ClínicaFIS \\ 
    \textbf{Notas}        &  Se presupone que se puede hacer uso de esta
    operación sin haber utilizado pedirCambioCitaRemoto, luego la fecha y la
    hora no tienen porque ser válidas.\\
    \textbf{Excepciones}    &
    \begin{itemizenomargins}
    \item[--] La fecha y la hora indicadas no son válidas.
    \item[--] No existe ninguna cita disponible para realizar el cambio.
    \end{itemizenomargins} \\
    \textbf{Salida}        &  infoCita = objeto de la clase Cita con la fecha y
    hora especificadas junto con el estado: pendiente.\\ 
    \textbf{Precondiciones}    &
    \begin{itemizenomargins}
    \item[--] La cita identificada por idCita existe en el sistema.
    \end{itemizenomargins} \\ 
    \textbf{Postcondiciones}  &
    \begin{itemizenomargins}
    \item[--] Se creó un objeto de tipo Cita.
    \item[--] Se creó un enlace entre el objeto de la clase Paciente con el que
      se encuentra relacionado el objeto de la clase Cita identificado por idCita y el objeto de la clase Cita.
    \item[--] Se creó un enlace entre el objeto de la clase Consulta con el que
      se encuentra relacionado el objeto de la clase Cita identificado por idCita y el objeto de la clase Cita.
    \item[--] Se destruyen los enlaces del objeto de la clase Cita identificado
      por idCita con las clases Paciente y Consulta
    \item[--] Se destruye el objeto de tipo Cita identificado por idCita.
      
    \end{itemizenomargins}

  \end{tabularx}
\end{table}

