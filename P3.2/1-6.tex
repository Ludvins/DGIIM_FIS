



\begin{table}[H]
\centering
	  \begin{tabularx}{\textwidth}{l|X}
    \textbf{Nombre}        & llamarSiguientePaciente(idSanitario) \\ 
    \textbf{Responsabilidad}  &  Se deberá avisar al siguiente paciente para que pase a la consulta del sanitario idSanitario \\ 
    \textbf{Tipo}       & ClinicaFIS  \\ 
    \textbf{Notas}        & Supondré que el estado Pendiente indica que el paciente está esperando, que el estado En espera indica que el paciente ha sido llamado ,que el estado Atendida significará que el paciente está siendo atendido y que el estado libre indica que la cita ha terminado.  \\ 
    \textbf{Excepciones}    & No hay siguiente paciente \\ 
    \textbf{Salida}        & Nombre del siguiente paciente\\ 
    \textbf{Precondiciones}    & La cita del paciente debe estar marcada como Pendiente \\ 
    \textbf{Postcondiciones}  &
    \begin{itemizenomargins}
   \item[--] El estado de la cita cambia a En Espera
   \end{itemizenomargins} \\ \\
  \end{tabularx}

\end{table}


\begin{table}[H]
\centering
	  \begin{tabularx}{\textwidth}{l|X}
    \textbf{Nombre}        & terminarConsulta(idPaciente,idSanitario) \\ 
    \textbf{Responsabilidad}  & Indicar que la cita actual ha terminado entre el paciente idPaciente y el sanitario idSanitario \\ 
    \textbf{Tipo}        &  ClinicaFIS\\ 
    \textbf{Notas}        &  \\ 
    \textbf{Excepciones}    &  \\ 
    \textbf{Salida}        &  \\ 
    \textbf{Precondiciones}    & El estado de la cita era Atendida  \\ 
    \textbf{Postcondiciones}  & \begin{itemizenomargins}
   \item[--] El estado de la cita cambia a Libre
   \end{itemizenomargins} \\ \\
  \end{tabularx}

\end{table}

\begin{table}[H]
\centering
	  \begin{tabularx}{\textwidth}{l|X}
    \textbf{Nombre}        & iniciarConsulta(idPaciente) \\ 
    \textbf{Responsabilidad}  & Dar comienzo a la cita y mostrar el historial clínico del paciente \\ 
    \textbf{Tipo}        &  ClinicaFIS\\ 
    \textbf{Notas}        &  \\ 
    \textbf{Excepciones}    & El paciente no tiene un historial clínico creado \\ 
    \textbf{Salida}        & Historial clínico del paciente \\ 
    \textbf{Precondiciones}    & El estado de la cita debe ser En Espera \\ 
    \textbf{Postcondiciones}  &  \begin{itemizenomargins}
   \item[--] El estado nuevo de la cita es Atendida
   \end{itemizenomargins} \\ \\
  \end{tabularx}

\end{table}


\begin{table}[H]
\centering
	  \begin{tabularx}{\textwidth}{l|X}
    \textbf{Nombre}        & registrarAlergia(idHC,tipoAlergia,textoExplicativo) \\ 
    \textbf{Responsabilidad}  & Introducir en el historial clínico idHC la alercia tipoAlergia con el comentario textoExplicativo \\ 
    \textbf{Tipo}        & ClinicaFIS \\ 
    \textbf{Notas}        &  \\ 
    \textbf{Excepciones}    & El idHC no existe o el tipoAlergia no es un dato válido \\ 
    \textbf{Salida}        &  \\ 
    \textbf{Precondiciones}    &  \\ 
    \textbf{Postcondiciones}  &  
    \begin{itemizenomargins}
   \item[--] Se crea un nuevo objeto de Alergia
   \item[--] Se asigna a ese objeto el tipo y el texto pasados como parámetros
   \item[--] Se crea un enlace entre HistoriaClínica y el objeto Alergia que se ha creado
   \end{itemizenomargins} \\ \\ 
  \end{tabularx}

\end{table}


\begin{table}[H]
\centering
	  \begin{tabularx}{\textwidth}{l|X}
    \textbf{Nombre}        & diagnosticar(idHC,codDiagnostico,textoExplicativo) \\ 
    \textbf{Responsabilidad}  & ClinicaFIS \\ 
    \textbf{Tipo}        &  ClinicaFIS\\ 
    \textbf{Notas}        &  Supondré que el sistema es capaz de dado el código de Diagnóstico indicar qué enfermedad sufre el paciente.\\ 
    \textbf{Excepciones}    & No existe ninguna enfermedad con ese código de Diagnóstico y ese texto explicativo \\ 
    \textbf{Salida}        & Enfermedad sufrida por el paciente  \\ 
    \textbf{Precondiciones}    &  \\ 
    \textbf{Postcondiciones}  & 
    \begin{itemizenomargins}
   \item[--] Se crea un objeto Diagnóstico
   \item[--] Se crea un objeto AnotaciónHC
   \item[--] Se asigna el comentario textoExplicativo y la fecha de la cita a los datos del objeto AnotaciónHC
   \item[--] Se crea un enlace entre el Diagnóstico y el AnotaciónHC creados.
   \item[--] Se crea un nuevo enlace entre la AnotaciónHC y la HistoriaClínica del paciente
   \end{itemizenomargins} \\ \\ 
  \end{tabularx}

\end{table}


\begin{table}[H]
\centering
	  \begin{tabularx}{\textwidth}{l|X}
    \textbf{Nombre}        & imponerTratamiento(idHC,tipoTratamiento,duracion,comentario) \\ 
    \textbf{Responsabilidad}  & Dotar a un paciente de un tratamiento para que se le sea impuesto  \\ 
    \textbf{Tipo}        & ClinicaFIS \\ 
    \textbf{Notas}        &  El tratamiento será acorde a la enfermedad que padezca\\ 
    \textbf{Excepciones}    & No Existe el tipo de tratamiento indicado \\ 
    \textbf{Salida}        &  \\ 
    \textbf{Precondiciones}    & El paciente no deberá tener ya en su Historia clínica ese tratamiento activo \\ 
    \textbf{Postcondiciones}  &  
    \begin{itemizenomargins}
   \item[--] Se crea un objeto Tratamiento
   \item[--] Se asigna el tipo y el perido indicados a los datos del tratamiento.
   \item[--] Se crea un objeto AnotacionHC con la fecha de la cita y el comentario pasados como parámetros
   \item[--] Se crea un enlace entre el objeto AnotacionHC y el objeto Tratamiento
   \item[--] Se crea un enlace entre la AnotacionHC y la HistoriaClinica del paciente.
   \end{itemizenomargins}\\ \\ 
  \end{tabularx}

\end{table}




