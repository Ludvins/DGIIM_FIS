 \begin{table}[H]
  \centering
  \begin{tabularx}{\textwidth}{l|X}
    \textbf{Nombre}        &  infoValoracion = valorarConsulta (idPaciente,
    idCita, puntuacion, comentario) \\ 
    \textbf{Responsabilidad}  & Crea una valoración por parte del paciente
    identificado por idPaciente sobre la cita identificada por idCita,
    utilizando el comentario y la puntuacion indicados.  \\ 
    \textbf{Tipo}        &  ClínicaFIS \\ 
    \textbf{Notas}        & No es necesario por parte del paciente indicar la
    fecha de la valoración, ya que el sistema dispondrá de ese dato. \\
    \textbf{Excepciones}    & El atributo ``estado'' de la cita es ATENDIDA.\\
    \textbf{Salida}        & infoValoracion\\ 
    \textbf{Precondiciones}    &
    \begin{itemizenomargins}
    \item[--] La cita identificada por idCita existe en el sistema.
    \item[--] El paciente identificado por idPaciente existe en el sistema.
    \end{itemizenomargins} \\ 
    \textbf{Postcondiciones}  &
    \begin{itemizenomargins}
    \item[--] Se creó un objeto de tipo Valoracion.
    \item[--] Se creó un enlace entre el objeto de la clase Valoracion con el que
      se encuentra relacionado el objeto de la clase Cita identificado por
      idCita.
      
    \end{itemizenomargins}

  \end{tabularx}
\end{table}
