
 \begin{table}[H]
  \centering
  \begin{tabularx}{\textwidth}{l|X}
    \textbf{Nombre}        &  listaPosiblesFechasHoras=pedirCitaRemoto(idPaciente, idConsulta) \\ 
    \textbf{Responsabilidad}  &  Crea una lista de posibles fechas y horas de citas para el paciente y la consulta identificados. \\ 
    \textbf{Tipo}        &  ClínicaFIS \\ 
    \textbf{Notas}        &  \\ 
    \textbf{Excepciones}    &
    \begin{itemizenomargins}
    \item[--] Si ya existe una cita en el sistema pedida por este mismo paciente.
    \item[--] No hay citas disponibles para la consulta identificada por idConsulta.\\
    \end{itemizenomargins} \\
    \textbf{Salida}        &  listaPosiblesFechasHoras = lista de \{fecha, hora\} para todos los objetos de la clase Cita que cumplan con los criterios de consulta. \\ 
    \textbf{Precondiciones}    &
    \begin{itemizenomargins}
    \item[--] El paciente identificado por idPaciente existe en el sistema.
    \item[--] La consulta identificada por idConsulta existe en el sistema.
    \end{itemizenomargins} \\ \\
    \textbf{Postcondiciones}  & Para cada elemento de listaPosiblesFechasHoras
     \begin{itemize}
    \item[--] Se creó un objeto de tipo Cita.
    \item[--] Se creó un enlace entre el objeto de la clase Paciente identificado
      por idPaciente y el objeto de la clase Cita.
    \item[--] Se creó un enlace entre el objeto de la clase Consulta identificado
      por idConsulta y el objeto de la clase Cita.
    \end{itemize}
  \end{tabularx}
\end{table}
 